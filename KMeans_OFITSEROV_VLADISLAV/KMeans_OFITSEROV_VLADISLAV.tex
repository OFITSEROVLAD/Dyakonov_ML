
    




    
\documentclass[11pt]{article}

    
    \usepackage[breakable]{tcolorbox}
    \tcbset{nobeforeafter} % prevents tcolorboxes being placing in paragraphs
    \usepackage{float}
    \floatplacement{figure}{H} % forces figures to be placed at the correct location
    
    \usepackage[T1]{fontenc}
    % Nicer default font (+ math font) than Computer Modern for most use cases
    \usepackage{mathpazo}

    % Basic figure setup, for now with no caption control since it's done
    % automatically by Pandoc (which extracts ![](path) syntax from Markdown).
    \usepackage{graphicx}
    % We will generate all images so they have a width \maxwidth. This means
    % that they will get their normal width if they fit onto the page, but
    % are scaled down if they would overflow the margins.
    \makeatletter
    \def\maxwidth{\ifdim\Gin@nat@width>\linewidth\linewidth
    \else\Gin@nat@width\fi}
    \makeatother
    \let\Oldincludegraphics\includegraphics
    % Set max figure width to be 80% of text width, for now hardcoded.
    \renewcommand{\includegraphics}[1]{\Oldincludegraphics[width=.8\maxwidth]{#1}}
    % Ensure that by default, figures have no caption (until we provide a
    % proper Figure object with a Caption API and a way to capture that
    % in the conversion process - todo).
    \usepackage{caption}
    \DeclareCaptionLabelFormat{nolabel}{}
    \captionsetup{labelformat=nolabel}

    \usepackage{adjustbox} % Used to constrain images to a maximum size 
    \usepackage{xcolor} % Allow colors to be defined
    \usepackage{enumerate} % Needed for markdown enumerations to work
    \usepackage{geometry} % Used to adjust the document margins
    \usepackage{amsmath} % Equations
    \usepackage{amssymb} % Equations
    \usepackage{textcomp} % defines textquotesingle
    % Hack from http://tex.stackexchange.com/a/47451/13684:
    \AtBeginDocument{%
        \def\PYZsq{\textquotesingle}% Upright quotes in Pygmentized code
    }
    \usepackage{upquote} % Upright quotes for verbatim code
    \usepackage{eurosym} % defines \euro
    \usepackage[mathletters]{ucs} % Extended unicode (utf-8) support
    \usepackage[utf8x]{inputenc} % Allow utf-8 characters in the tex document
    \usepackage{fancyvrb} % verbatim replacement that allows latex
    \usepackage{grffile} % extends the file name processing of package graphics 
                         % to support a larger range 
    % The hyperref package gives us a pdf with properly built
    % internal navigation ('pdf bookmarks' for the table of contents,
    % internal cross-reference links, web links for URLs, etc.)
    \usepackage{hyperref}
    \usepackage{longtable} % longtable support required by pandoc >1.10
    \usepackage{booktabs}  % table support for pandoc > 1.12.2
    \usepackage[inline]{enumitem} % IRkernel/repr support (it uses the enumerate* environment)
    \usepackage[normalem]{ulem} % ulem is needed to support strikethroughs (\sout)
                                % normalem makes italics be italics, not underlines
    \usepackage{mathrsfs}
    

    
    % Colors for the hyperref package
    \definecolor{urlcolor}{rgb}{0,.145,.698}
    \definecolor{linkcolor}{rgb}{.71,0.21,0.01}
    \definecolor{citecolor}{rgb}{.12,.54,.11}

    % ANSI colors
    \definecolor{ansi-black}{HTML}{3E424D}
    \definecolor{ansi-black-intense}{HTML}{282C36}
    \definecolor{ansi-red}{HTML}{E75C58}
    \definecolor{ansi-red-intense}{HTML}{B22B31}
    \definecolor{ansi-green}{HTML}{00A250}
    \definecolor{ansi-green-intense}{HTML}{007427}
    \definecolor{ansi-yellow}{HTML}{DDB62B}
    \definecolor{ansi-yellow-intense}{HTML}{B27D12}
    \definecolor{ansi-blue}{HTML}{208FFB}
    \definecolor{ansi-blue-intense}{HTML}{0065CA}
    \definecolor{ansi-magenta}{HTML}{D160C4}
    \definecolor{ansi-magenta-intense}{HTML}{A03196}
    \definecolor{ansi-cyan}{HTML}{60C6C8}
    \definecolor{ansi-cyan-intense}{HTML}{258F8F}
    \definecolor{ansi-white}{HTML}{C5C1B4}
    \definecolor{ansi-white-intense}{HTML}{A1A6B2}
    \definecolor{ansi-default-inverse-fg}{HTML}{FFFFFF}
    \definecolor{ansi-default-inverse-bg}{HTML}{000000}

    % commands and environments needed by pandoc snippets
    % extracted from the output of `pandoc -s`
    \providecommand{\tightlist}{%
      \setlength{\itemsep}{0pt}\setlength{\parskip}{0pt}}
    \DefineVerbatimEnvironment{Highlighting}{Verbatim}{commandchars=\\\{\}}
    % Add ',fontsize=\small' for more characters per line
    \newenvironment{Shaded}{}{}
    \newcommand{\KeywordTok}[1]{\textcolor[rgb]{0.00,0.44,0.13}{\textbf{{#1}}}}
    \newcommand{\DataTypeTok}[1]{\textcolor[rgb]{0.56,0.13,0.00}{{#1}}}
    \newcommand{\DecValTok}[1]{\textcolor[rgb]{0.25,0.63,0.44}{{#1}}}
    \newcommand{\BaseNTok}[1]{\textcolor[rgb]{0.25,0.63,0.44}{{#1}}}
    \newcommand{\FloatTok}[1]{\textcolor[rgb]{0.25,0.63,0.44}{{#1}}}
    \newcommand{\CharTok}[1]{\textcolor[rgb]{0.25,0.44,0.63}{{#1}}}
    \newcommand{\StringTok}[1]{\textcolor[rgb]{0.25,0.44,0.63}{{#1}}}
    \newcommand{\CommentTok}[1]{\textcolor[rgb]{0.38,0.63,0.69}{\textit{{#1}}}}
    \newcommand{\OtherTok}[1]{\textcolor[rgb]{0.00,0.44,0.13}{{#1}}}
    \newcommand{\AlertTok}[1]{\textcolor[rgb]{1.00,0.00,0.00}{\textbf{{#1}}}}
    \newcommand{\FunctionTok}[1]{\textcolor[rgb]{0.02,0.16,0.49}{{#1}}}
    \newcommand{\RegionMarkerTok}[1]{{#1}}
    \newcommand{\ErrorTok}[1]{\textcolor[rgb]{1.00,0.00,0.00}{\textbf{{#1}}}}
    \newcommand{\NormalTok}[1]{{#1}}
    
    % Additional commands for more recent versions of Pandoc
    \newcommand{\ConstantTok}[1]{\textcolor[rgb]{0.53,0.00,0.00}{{#1}}}
    \newcommand{\SpecialCharTok}[1]{\textcolor[rgb]{0.25,0.44,0.63}{{#1}}}
    \newcommand{\VerbatimStringTok}[1]{\textcolor[rgb]{0.25,0.44,0.63}{{#1}}}
    \newcommand{\SpecialStringTok}[1]{\textcolor[rgb]{0.73,0.40,0.53}{{#1}}}
    \newcommand{\ImportTok}[1]{{#1}}
    \newcommand{\DocumentationTok}[1]{\textcolor[rgb]{0.73,0.13,0.13}{\textit{{#1}}}}
    \newcommand{\AnnotationTok}[1]{\textcolor[rgb]{0.38,0.63,0.69}{\textbf{\textit{{#1}}}}}
    \newcommand{\CommentVarTok}[1]{\textcolor[rgb]{0.38,0.63,0.69}{\textbf{\textit{{#1}}}}}
    \newcommand{\VariableTok}[1]{\textcolor[rgb]{0.10,0.09,0.49}{{#1}}}
    \newcommand{\ControlFlowTok}[1]{\textcolor[rgb]{0.00,0.44,0.13}{\textbf{{#1}}}}
    \newcommand{\OperatorTok}[1]{\textcolor[rgb]{0.40,0.40,0.40}{{#1}}}
    \newcommand{\BuiltInTok}[1]{{#1}}
    \newcommand{\ExtensionTok}[1]{{#1}}
    \newcommand{\PreprocessorTok}[1]{\textcolor[rgb]{0.74,0.48,0.00}{{#1}}}
    \newcommand{\AttributeTok}[1]{\textcolor[rgb]{0.49,0.56,0.16}{{#1}}}
    \newcommand{\InformationTok}[1]{\textcolor[rgb]{0.38,0.63,0.69}{\textbf{\textit{{#1}}}}}
    \newcommand{\WarningTok}[1]{\textcolor[rgb]{0.38,0.63,0.69}{\textbf{\textit{{#1}}}}}
    
    
    % Define a nice break command that doesn't care if a line doesn't already
    % exist.
    \def\br{\hspace*{\fill} \\* }
    % Math Jax compatibility definitions
    \def\gt{>}
    \def\lt{<}
    \let\Oldtex\TeX
    \let\Oldlatex\LaTeX
    \renewcommand{\TeX}{\textrm{\Oldtex}}
    \renewcommand{\LaTeX}{\textrm{\Oldlatex}}
    % Document parameters
    % Document title
    \title{KMeans\_OFITSEROV\_VLADISLAV}
    
    
    
    
    
% Pygments definitions
\makeatletter
\def\PY@reset{\let\PY@it=\relax \let\PY@bf=\relax%
    \let\PY@ul=\relax \let\PY@tc=\relax%
    \let\PY@bc=\relax \let\PY@ff=\relax}
\def\PY@tok#1{\csname PY@tok@#1\endcsname}
\def\PY@toks#1+{\ifx\relax#1\empty\else%
    \PY@tok{#1}\expandafter\PY@toks\fi}
\def\PY@do#1{\PY@bc{\PY@tc{\PY@ul{%
    \PY@it{\PY@bf{\PY@ff{#1}}}}}}}
\def\PY#1#2{\PY@reset\PY@toks#1+\relax+\PY@do{#2}}

\expandafter\def\csname PY@tok@w\endcsname{\def\PY@tc##1{\textcolor[rgb]{0.73,0.73,0.73}{##1}}}
\expandafter\def\csname PY@tok@c\endcsname{\let\PY@it=\textit\def\PY@tc##1{\textcolor[rgb]{0.25,0.50,0.50}{##1}}}
\expandafter\def\csname PY@tok@cp\endcsname{\def\PY@tc##1{\textcolor[rgb]{0.74,0.48,0.00}{##1}}}
\expandafter\def\csname PY@tok@k\endcsname{\let\PY@bf=\textbf\def\PY@tc##1{\textcolor[rgb]{0.00,0.50,0.00}{##1}}}
\expandafter\def\csname PY@tok@kp\endcsname{\def\PY@tc##1{\textcolor[rgb]{0.00,0.50,0.00}{##1}}}
\expandafter\def\csname PY@tok@kt\endcsname{\def\PY@tc##1{\textcolor[rgb]{0.69,0.00,0.25}{##1}}}
\expandafter\def\csname PY@tok@o\endcsname{\def\PY@tc##1{\textcolor[rgb]{0.40,0.40,0.40}{##1}}}
\expandafter\def\csname PY@tok@ow\endcsname{\let\PY@bf=\textbf\def\PY@tc##1{\textcolor[rgb]{0.67,0.13,1.00}{##1}}}
\expandafter\def\csname PY@tok@nb\endcsname{\def\PY@tc##1{\textcolor[rgb]{0.00,0.50,0.00}{##1}}}
\expandafter\def\csname PY@tok@nf\endcsname{\def\PY@tc##1{\textcolor[rgb]{0.00,0.00,1.00}{##1}}}
\expandafter\def\csname PY@tok@nc\endcsname{\let\PY@bf=\textbf\def\PY@tc##1{\textcolor[rgb]{0.00,0.00,1.00}{##1}}}
\expandafter\def\csname PY@tok@nn\endcsname{\let\PY@bf=\textbf\def\PY@tc##1{\textcolor[rgb]{0.00,0.00,1.00}{##1}}}
\expandafter\def\csname PY@tok@ne\endcsname{\let\PY@bf=\textbf\def\PY@tc##1{\textcolor[rgb]{0.82,0.25,0.23}{##1}}}
\expandafter\def\csname PY@tok@nv\endcsname{\def\PY@tc##1{\textcolor[rgb]{0.10,0.09,0.49}{##1}}}
\expandafter\def\csname PY@tok@no\endcsname{\def\PY@tc##1{\textcolor[rgb]{0.53,0.00,0.00}{##1}}}
\expandafter\def\csname PY@tok@nl\endcsname{\def\PY@tc##1{\textcolor[rgb]{0.63,0.63,0.00}{##1}}}
\expandafter\def\csname PY@tok@ni\endcsname{\let\PY@bf=\textbf\def\PY@tc##1{\textcolor[rgb]{0.60,0.60,0.60}{##1}}}
\expandafter\def\csname PY@tok@na\endcsname{\def\PY@tc##1{\textcolor[rgb]{0.49,0.56,0.16}{##1}}}
\expandafter\def\csname PY@tok@nt\endcsname{\let\PY@bf=\textbf\def\PY@tc##1{\textcolor[rgb]{0.00,0.50,0.00}{##1}}}
\expandafter\def\csname PY@tok@nd\endcsname{\def\PY@tc##1{\textcolor[rgb]{0.67,0.13,1.00}{##1}}}
\expandafter\def\csname PY@tok@s\endcsname{\def\PY@tc##1{\textcolor[rgb]{0.73,0.13,0.13}{##1}}}
\expandafter\def\csname PY@tok@sd\endcsname{\let\PY@it=\textit\def\PY@tc##1{\textcolor[rgb]{0.73,0.13,0.13}{##1}}}
\expandafter\def\csname PY@tok@si\endcsname{\let\PY@bf=\textbf\def\PY@tc##1{\textcolor[rgb]{0.73,0.40,0.53}{##1}}}
\expandafter\def\csname PY@tok@se\endcsname{\let\PY@bf=\textbf\def\PY@tc##1{\textcolor[rgb]{0.73,0.40,0.13}{##1}}}
\expandafter\def\csname PY@tok@sr\endcsname{\def\PY@tc##1{\textcolor[rgb]{0.73,0.40,0.53}{##1}}}
\expandafter\def\csname PY@tok@ss\endcsname{\def\PY@tc##1{\textcolor[rgb]{0.10,0.09,0.49}{##1}}}
\expandafter\def\csname PY@tok@sx\endcsname{\def\PY@tc##1{\textcolor[rgb]{0.00,0.50,0.00}{##1}}}
\expandafter\def\csname PY@tok@m\endcsname{\def\PY@tc##1{\textcolor[rgb]{0.40,0.40,0.40}{##1}}}
\expandafter\def\csname PY@tok@gh\endcsname{\let\PY@bf=\textbf\def\PY@tc##1{\textcolor[rgb]{0.00,0.00,0.50}{##1}}}
\expandafter\def\csname PY@tok@gu\endcsname{\let\PY@bf=\textbf\def\PY@tc##1{\textcolor[rgb]{0.50,0.00,0.50}{##1}}}
\expandafter\def\csname PY@tok@gd\endcsname{\def\PY@tc##1{\textcolor[rgb]{0.63,0.00,0.00}{##1}}}
\expandafter\def\csname PY@tok@gi\endcsname{\def\PY@tc##1{\textcolor[rgb]{0.00,0.63,0.00}{##1}}}
\expandafter\def\csname PY@tok@gr\endcsname{\def\PY@tc##1{\textcolor[rgb]{1.00,0.00,0.00}{##1}}}
\expandafter\def\csname PY@tok@ge\endcsname{\let\PY@it=\textit}
\expandafter\def\csname PY@tok@gs\endcsname{\let\PY@bf=\textbf}
\expandafter\def\csname PY@tok@gp\endcsname{\let\PY@bf=\textbf\def\PY@tc##1{\textcolor[rgb]{0.00,0.00,0.50}{##1}}}
\expandafter\def\csname PY@tok@go\endcsname{\def\PY@tc##1{\textcolor[rgb]{0.53,0.53,0.53}{##1}}}
\expandafter\def\csname PY@tok@gt\endcsname{\def\PY@tc##1{\textcolor[rgb]{0.00,0.27,0.87}{##1}}}
\expandafter\def\csname PY@tok@err\endcsname{\def\PY@bc##1{\setlength{\fboxsep}{0pt}\fcolorbox[rgb]{1.00,0.00,0.00}{1,1,1}{\strut ##1}}}
\expandafter\def\csname PY@tok@kc\endcsname{\let\PY@bf=\textbf\def\PY@tc##1{\textcolor[rgb]{0.00,0.50,0.00}{##1}}}
\expandafter\def\csname PY@tok@kd\endcsname{\let\PY@bf=\textbf\def\PY@tc##1{\textcolor[rgb]{0.00,0.50,0.00}{##1}}}
\expandafter\def\csname PY@tok@kn\endcsname{\let\PY@bf=\textbf\def\PY@tc##1{\textcolor[rgb]{0.00,0.50,0.00}{##1}}}
\expandafter\def\csname PY@tok@kr\endcsname{\let\PY@bf=\textbf\def\PY@tc##1{\textcolor[rgb]{0.00,0.50,0.00}{##1}}}
\expandafter\def\csname PY@tok@bp\endcsname{\def\PY@tc##1{\textcolor[rgb]{0.00,0.50,0.00}{##1}}}
\expandafter\def\csname PY@tok@fm\endcsname{\def\PY@tc##1{\textcolor[rgb]{0.00,0.00,1.00}{##1}}}
\expandafter\def\csname PY@tok@vc\endcsname{\def\PY@tc##1{\textcolor[rgb]{0.10,0.09,0.49}{##1}}}
\expandafter\def\csname PY@tok@vg\endcsname{\def\PY@tc##1{\textcolor[rgb]{0.10,0.09,0.49}{##1}}}
\expandafter\def\csname PY@tok@vi\endcsname{\def\PY@tc##1{\textcolor[rgb]{0.10,0.09,0.49}{##1}}}
\expandafter\def\csname PY@tok@vm\endcsname{\def\PY@tc##1{\textcolor[rgb]{0.10,0.09,0.49}{##1}}}
\expandafter\def\csname PY@tok@sa\endcsname{\def\PY@tc##1{\textcolor[rgb]{0.73,0.13,0.13}{##1}}}
\expandafter\def\csname PY@tok@sb\endcsname{\def\PY@tc##1{\textcolor[rgb]{0.73,0.13,0.13}{##1}}}
\expandafter\def\csname PY@tok@sc\endcsname{\def\PY@tc##1{\textcolor[rgb]{0.73,0.13,0.13}{##1}}}
\expandafter\def\csname PY@tok@dl\endcsname{\def\PY@tc##1{\textcolor[rgb]{0.73,0.13,0.13}{##1}}}
\expandafter\def\csname PY@tok@s2\endcsname{\def\PY@tc##1{\textcolor[rgb]{0.73,0.13,0.13}{##1}}}
\expandafter\def\csname PY@tok@sh\endcsname{\def\PY@tc##1{\textcolor[rgb]{0.73,0.13,0.13}{##1}}}
\expandafter\def\csname PY@tok@s1\endcsname{\def\PY@tc##1{\textcolor[rgb]{0.73,0.13,0.13}{##1}}}
\expandafter\def\csname PY@tok@mb\endcsname{\def\PY@tc##1{\textcolor[rgb]{0.40,0.40,0.40}{##1}}}
\expandafter\def\csname PY@tok@mf\endcsname{\def\PY@tc##1{\textcolor[rgb]{0.40,0.40,0.40}{##1}}}
\expandafter\def\csname PY@tok@mh\endcsname{\def\PY@tc##1{\textcolor[rgb]{0.40,0.40,0.40}{##1}}}
\expandafter\def\csname PY@tok@mi\endcsname{\def\PY@tc##1{\textcolor[rgb]{0.40,0.40,0.40}{##1}}}
\expandafter\def\csname PY@tok@il\endcsname{\def\PY@tc##1{\textcolor[rgb]{0.40,0.40,0.40}{##1}}}
\expandafter\def\csname PY@tok@mo\endcsname{\def\PY@tc##1{\textcolor[rgb]{0.40,0.40,0.40}{##1}}}
\expandafter\def\csname PY@tok@ch\endcsname{\let\PY@it=\textit\def\PY@tc##1{\textcolor[rgb]{0.25,0.50,0.50}{##1}}}
\expandafter\def\csname PY@tok@cm\endcsname{\let\PY@it=\textit\def\PY@tc##1{\textcolor[rgb]{0.25,0.50,0.50}{##1}}}
\expandafter\def\csname PY@tok@cpf\endcsname{\let\PY@it=\textit\def\PY@tc##1{\textcolor[rgb]{0.25,0.50,0.50}{##1}}}
\expandafter\def\csname PY@tok@c1\endcsname{\let\PY@it=\textit\def\PY@tc##1{\textcolor[rgb]{0.25,0.50,0.50}{##1}}}
\expandafter\def\csname PY@tok@cs\endcsname{\let\PY@it=\textit\def\PY@tc##1{\textcolor[rgb]{0.25,0.50,0.50}{##1}}}

\def\PYZbs{\char`\\}
\def\PYZus{\char`\_}
\def\PYZob{\char`\{}
\def\PYZcb{\char`\}}
\def\PYZca{\char`\^}
\def\PYZam{\char`\&}
\def\PYZlt{\char`\<}
\def\PYZgt{\char`\>}
\def\PYZsh{\char`\#}
\def\PYZpc{\char`\%}
\def\PYZdl{\char`\$}
\def\PYZhy{\char`\-}
\def\PYZsq{\char`\'}
\def\PYZdq{\char`\"}
\def\PYZti{\char`\~}
% for compatibility with earlier versions
\def\PYZat{@}
\def\PYZlb{[}
\def\PYZrb{]}
\makeatother


    % For linebreaks inside Verbatim environment from package fancyvrb. 
    \makeatletter
        \newbox\Wrappedcontinuationbox 
        \newbox\Wrappedvisiblespacebox 
        \newcommand*\Wrappedvisiblespace {\textcolor{red}{\textvisiblespace}} 
        \newcommand*\Wrappedcontinuationsymbol {\textcolor{red}{\llap{\tiny$\m@th\hookrightarrow$}}} 
        \newcommand*\Wrappedcontinuationindent {3ex } 
        \newcommand*\Wrappedafterbreak {\kern\Wrappedcontinuationindent\copy\Wrappedcontinuationbox} 
        % Take advantage of the already applied Pygments mark-up to insert 
        % potential linebreaks for TeX processing. 
        %        {, <, #, %, $, ' and ": go to next line. 
        %        _, }, ^, &, >, - and ~: stay at end of broken line. 
        % Use of \textquotesingle for straight quote. 
        \newcommand*\Wrappedbreaksatspecials {% 
            \def\PYGZus{\discretionary{\char`\_}{\Wrappedafterbreak}{\char`\_}}% 
            \def\PYGZob{\discretionary{}{\Wrappedafterbreak\char`\{}{\char`\{}}% 
            \def\PYGZcb{\discretionary{\char`\}}{\Wrappedafterbreak}{\char`\}}}% 
            \def\PYGZca{\discretionary{\char`\^}{\Wrappedafterbreak}{\char`\^}}% 
            \def\PYGZam{\discretionary{\char`\&}{\Wrappedafterbreak}{\char`\&}}% 
            \def\PYGZlt{\discretionary{}{\Wrappedafterbreak\char`\<}{\char`\<}}% 
            \def\PYGZgt{\discretionary{\char`\>}{\Wrappedafterbreak}{\char`\>}}% 
            \def\PYGZsh{\discretionary{}{\Wrappedafterbreak\char`\#}{\char`\#}}% 
            \def\PYGZpc{\discretionary{}{\Wrappedafterbreak\char`\%}{\char`\%}}% 
            \def\PYGZdl{\discretionary{}{\Wrappedafterbreak\char`\$}{\char`\$}}% 
            \def\PYGZhy{\discretionary{\char`\-}{\Wrappedafterbreak}{\char`\-}}% 
            \def\PYGZsq{\discretionary{}{\Wrappedafterbreak\textquotesingle}{\textquotesingle}}% 
            \def\PYGZdq{\discretionary{}{\Wrappedafterbreak\char`\"}{\char`\"}}% 
            \def\PYGZti{\discretionary{\char`\~}{\Wrappedafterbreak}{\char`\~}}% 
        } 
        % Some characters . , ; ? ! / are not pygmentized. 
        % This macro makes them "active" and they will insert potential linebreaks 
        \newcommand*\Wrappedbreaksatpunct {% 
            \lccode`\~`\.\lowercase{\def~}{\discretionary{\hbox{\char`\.}}{\Wrappedafterbreak}{\hbox{\char`\.}}}% 
            \lccode`\~`\,\lowercase{\def~}{\discretionary{\hbox{\char`\,}}{\Wrappedafterbreak}{\hbox{\char`\,}}}% 
            \lccode`\~`\;\lowercase{\def~}{\discretionary{\hbox{\char`\;}}{\Wrappedafterbreak}{\hbox{\char`\;}}}% 
            \lccode`\~`\:\lowercase{\def~}{\discretionary{\hbox{\char`\:}}{\Wrappedafterbreak}{\hbox{\char`\:}}}% 
            \lccode`\~`\?\lowercase{\def~}{\discretionary{\hbox{\char`\?}}{\Wrappedafterbreak}{\hbox{\char`\?}}}% 
            \lccode`\~`\!\lowercase{\def~}{\discretionary{\hbox{\char`\!}}{\Wrappedafterbreak}{\hbox{\char`\!}}}% 
            \lccode`\~`\/\lowercase{\def~}{\discretionary{\hbox{\char`\/}}{\Wrappedafterbreak}{\hbox{\char`\/}}}% 
            \catcode`\.\active
            \catcode`\,\active 
            \catcode`\;\active
            \catcode`\:\active
            \catcode`\?\active
            \catcode`\!\active
            \catcode`\/\active 
            \lccode`\~`\~ 	
        }
    \makeatother

    \let\OriginalVerbatim=\Verbatim
    \makeatletter
    \renewcommand{\Verbatim}[1][1]{%
        %\parskip\z@skip
        \sbox\Wrappedcontinuationbox {\Wrappedcontinuationsymbol}%
        \sbox\Wrappedvisiblespacebox {\FV@SetupFont\Wrappedvisiblespace}%
        \def\FancyVerbFormatLine ##1{\hsize\linewidth
            \vtop{\raggedright\hyphenpenalty\z@\exhyphenpenalty\z@
                \doublehyphendemerits\z@\finalhyphendemerits\z@
                \strut ##1\strut}%
        }%
        % If the linebreak is at a space, the latter will be displayed as visible
        % space at end of first line, and a continuation symbol starts next line.
        % Stretch/shrink are however usually zero for typewriter font.
        \def\FV@Space {%
            \nobreak\hskip\z@ plus\fontdimen3\font minus\fontdimen4\font
            \discretionary{\copy\Wrappedvisiblespacebox}{\Wrappedafterbreak}
            {\kern\fontdimen2\font}%
        }%
        
        % Allow breaks at special characters using \PYG... macros.
        \Wrappedbreaksatspecials
        % Breaks at punctuation characters . , ; ? ! and / need catcode=\active 	
        \OriginalVerbatim[#1,codes*=\Wrappedbreaksatpunct]%
    }
    \makeatother

    % Exact colors from NB
    \definecolor{incolor}{HTML}{303F9F}
    \definecolor{outcolor}{HTML}{D84315}
    \definecolor{cellborder}{HTML}{CFCFCF}
    \definecolor{cellbackground}{HTML}{F7F7F7}
    
    % prompt
    \newcommand{\prompt}[4]{
        \llap{{\color{#2}[#3]: #4}}\vspace{-1.25em}
    }
    

    
    % Prevent overflowing lines due to hard-to-break entities
    \sloppy 
    % Setup hyperref package
    \hypersetup{
      breaklinks=true,  % so long urls are correctly broken across lines
      colorlinks=true,
      urlcolor=urlcolor,
      linkcolor=linkcolor,
      citecolor=citecolor,
      }
    % Slightly bigger margins than the latex defaults
    
    \geometry{verbose,tmargin=1in,bmargin=1in,lmargin=1in,rmargin=1in}
    
    

    \begin{document}
    
    
    \maketitle
    
    

    
    \hypertarget{kmeans}{%
\section{kMeans}\label{kmeans}}

\hypertarget{ux43eux444ux438ux446ux435ux440ux43eux432-ux432ux43bux430ux434ux438ux441ux43bux430ux432}{%
\subsubsection{Офицеров
Владислав}\label{ux43eux444ux438ux446ux435ux440ux43eux432-ux432ux43bux430ux434ux438ux441ux43bux430ux432}}

    \begin{tcolorbox}[breakable, size=fbox, boxrule=1pt, pad at break*=1mm,colback=cellbackground, colframe=cellborder]
\prompt{In}{incolor}{137}{\hspace{4pt}}
\begin{Verbatim}[commandchars=\\\{\}]
\PY{k+kn}{import} \PY{n+nn}{numpy} \PY{k}{as} \PY{n+nn}{np}
\PY{k+kn}{import} \PY{n+nn}{matplotlib}\PY{n+nn}{.}\PY{n+nn}{pyplot} \PY{k}{as} \PY{n+nn}{plt}
\PY{k+kn}{import} \PY{n+nn}{copy}
\PY{k+kn}{from} \PY{n+nn}{sklearn}\PY{n+nn}{.}\PY{n+nn}{metrics} \PY{k}{import} \PY{n}{silhouette\PYZus{}score}
\PY{k+kn}{from} \PY{n+nn}{sklearn}\PY{n+nn}{.}\PY{n+nn}{metrics}\PY{n+nn}{.}\PY{n+nn}{pairwise} \PY{k}{import} \PY{n}{pairwise\PYZus{}distances}
\PY{k+kn}{from} \PY{n+nn}{sklearn}\PY{n+nn}{.}\PY{n+nn}{datasets} \PY{k}{import} \PY{n}{make\PYZus{}blobs}
\PY{o}{\PYZpc{}}\PY{k}{matplotlib} inline
\end{Verbatim}
\end{tcolorbox}

    \begin{tcolorbox}[breakable, size=fbox, boxrule=1pt, pad at break*=1mm,colback=cellbackground, colframe=cellborder]
\prompt{In}{incolor}{162}{\hspace{4pt}}
\begin{Verbatim}[commandchars=\\\{\}]
\PY{k}{def} \PY{n+nf}{myKMeans}\PY{p}{(}\PY{n}{k}\PY{p}{,} \PY{n}{data}\PY{p}{,} \PY{n}{y}\PY{p}{,} \PY{n}{s}\PY{p}{)}\PY{p}{:}
\PY{c+c1}{\PYZsh{}     Функция показа   }
    \PY{k}{def} \PY{n+nf}{pltshow}\PY{p}{(}\PY{p}{)}\PY{p}{:}
        \PY{n}{plt}\PY{o}{.}\PY{n}{xlim}\PY{p}{(}\PY{n}{data}\PY{p}{[}\PY{p}{:}\PY{p}{,}\PY{l+m+mi}{0}\PY{p}{]}\PY{o}{.}\PY{n}{min}\PY{p}{(}\PY{p}{)}\PY{p}{,}\PY{n}{data}\PY{p}{[}\PY{p}{:}\PY{p}{,}\PY{l+m+mi}{0}\PY{p}{]}\PY{o}{.}\PY{n}{max}\PY{p}{(}\PY{p}{)}\PY{p}{)}
        \PY{n}{plt}\PY{o}{.}\PY{n}{ylim}\PY{p}{(}\PY{n}{data}\PY{p}{[}\PY{p}{:}\PY{p}{,}\PY{l+m+mi}{1}\PY{p}{]}\PY{o}{.}\PY{n}{min}\PY{p}{(}\PY{p}{)}\PY{p}{,}\PY{n}{data}\PY{p}{[}\PY{p}{:}\PY{p}{,}\PY{l+m+mi}{1}\PY{p}{]}\PY{o}{.}\PY{n}{max}\PY{p}{(}\PY{p}{)}\PY{p}{)}
        \PY{n}{plt}\PY{o}{.}\PY{n}{show}\PY{p}{(}\PY{p}{)}

\PY{c+c1}{\PYZsh{}     Добавление на график центроидов      }
    \PY{k}{def} \PY{n+nf}{cent}\PY{p}{(}\PY{p}{)}\PY{p}{:}
        \PY{n}{fig} \PY{o}{=} \PY{n}{plt}\PY{o}{.}\PY{n}{figure}\PY{p}{(}\PY{n}{figsize}\PY{o}{=}\PY{p}{(}\PY{l+m+mi}{7}\PY{p}{,} \PY{l+m+mi}{7}\PY{p}{)}\PY{p}{)}
        \PY{k}{for} \PY{n}{i} \PY{o+ow}{in} \PY{n}{centroids}\PY{o}{.}\PY{n}{keys}\PY{p}{(}\PY{p}{)}\PY{p}{:}
            \PY{n}{plt}\PY{o}{.}\PY{n}{scatter}\PY{p}{(}\PY{o}{*}\PY{n}{centroids}\PY{p}{[}\PY{n}{i}\PY{p}{]}\PY{p}{,} \PY{n}{color}\PY{o}{=}\PY{n}{colmap}\PY{p}{[}\PY{n}{i} \PY{o}{\PYZpc{}} \PY{l+m+mi}{6} \PY{o}{+} \PY{l+m+mi}{1}\PY{p}{]}\PY{p}{,} \PY{n}{marker} \PY{o}{=} \PY{l+s+s1}{\PYZsq{}}\PY{l+s+s1}{*}\PY{l+s+s1}{\PYZsq{}}\PY{p}{)}

\PY{c+c1}{\PYZsh{}     Добавление на график объектов }
    \PY{k}{def} \PY{n+nf}{poin}\PY{p}{(}\PY{p}{)}\PY{p}{:}
        \PY{n}{plt}\PY{o}{.}\PY{n}{scatter}\PY{p}{(}\PY{n}{data}\PY{p}{[}\PY{p}{:}\PY{p}{,}\PY{l+m+mi}{0}\PY{p}{]}\PY{p}{,} \PY{n}{data}\PY{p}{[}\PY{p}{:}\PY{p}{,}\PY{l+m+mi}{1}\PY{p}{]}\PY{p}{,} \PY{n}{color}\PY{o}{=}\PY{n}{cl}\PY{p}{,} \PY{n}{alpha}\PY{o}{=}\PY{l+m+mf}{0.6}\PY{p}{,} \PY{n}{edgecolors}\PY{o}{=}\PY{n}{cl}\PY{p}{,} \PY{n}{marker} \PY{o}{=} \PY{l+s+s1}{\PYZsq{}}\PY{l+s+s1}{.}\PY{l+s+s1}{\PYZsq{}}\PY{p}{)}
        
\PY{c+c1}{\PYZsh{}    Вспомогатильные переменные     }
    \PY{n}{np}\PY{o}{.}\PY{n}{random}\PY{o}{.}\PY{n}{seed}\PY{p}{(}\PY{l+m+mi}{123}\PY{p}{)}
\PY{c+c1}{\PYZsh{}    Количество объектов}
    \PY{n}{n} \PY{o}{=} \PY{n}{data}\PY{o}{.}\PY{n}{shape}\PY{p}{[}\PY{l+m+mi}{0}\PY{p}{]}
\PY{c+c1}{\PYZsh{}     Количество фичей}
    \PY{n}{m} \PY{o}{=} \PY{n}{data}\PY{o}{.}\PY{n}{shape}\PY{p}{[}\PY{l+m+mi}{1}\PY{p}{]}
\PY{c+c1}{\PYZsh{}     Цветовая гамма:}
    \PY{n}{colmap} \PY{o}{=} \PY{p}{\PYZob{}}\PY{l+m+mi}{1}\PY{p}{:} \PY{l+s+s1}{\PYZsq{}}\PY{l+s+s1}{r}\PY{l+s+s1}{\PYZsq{}}\PY{p}{,} \PY{l+m+mi}{2}\PY{p}{:} \PY{l+s+s1}{\PYZsq{}}\PY{l+s+s1}{g}\PY{l+s+s1}{\PYZsq{}}\PY{p}{,} \PY{l+m+mi}{3}\PY{p}{:} \PY{l+s+s1}{\PYZsq{}}\PY{l+s+s1}{b}\PY{l+s+s1}{\PYZsq{}}\PY{p}{,} \PY{l+m+mi}{4}\PY{p}{:} \PY{l+s+s1}{\PYZsq{}}\PY{l+s+s1}{c}\PY{l+s+s1}{\PYZsq{}}\PY{p}{,} \PY{l+m+mi}{5}\PY{p}{:} \PY{l+s+s1}{\PYZsq{}}\PY{l+s+s1}{m}\PY{l+s+s1}{\PYZsq{}}\PY{p}{,} \PY{l+m+mi}{6}\PY{p}{:} \PY{l+s+s1}{\PYZsq{}}\PY{l+s+s1}{y}\PY{l+s+s1}{\PYZsq{}}\PY{p}{\PYZcb{}} 
\PY{c+c1}{\PYZsh{}     Массив для цвета кажого объекта}
    \PY{n}{cl} \PY{o}{=} \PY{n}{np}\PY{o}{.}\PY{n}{full}\PY{p}{(}\PY{n}{n}\PY{p}{,}\PY{l+s+s1}{\PYZsq{}}\PY{l+s+s1}{ }\PY{l+s+s1}{\PYZsq{}}\PY{p}{)}


\PY{c+c1}{\PYZsh{}     Визуализация данных}
    \PY{k}{if} \PY{p}{(}\PY{n}{y}\PY{p}{)}\PY{p}{:}
        \PY{n+nb}{print}\PY{p}{(}\PY{l+s+s1}{\PYZsq{}}\PY{l+s+se}{\PYZbs{}n}\PY{l+s+se}{\PYZbs{}n}\PY{l+s+s1}{NUMBER OF CLUSTERS:}\PY{l+s+s1}{\PYZsq{}}\PY{p}{,} \PY{n}{k}\PY{p}{,}\PY{l+s+s1}{\PYZsq{}}\PY{l+s+se}{\PYZbs{}n}\PY{l+s+se}{\PYZbs{}n}\PY{l+s+s1}{\PYZsq{}}\PY{p}{)}
        \PY{n+nb}{print}\PY{p}{(}\PY{l+s+s1}{\PYZsq{}}\PY{l+s+se}{\PYZbs{}n}\PY{l+s+se}{\PYZbs{}n}\PY{l+s+s1}{Визуализация данных}\PY{l+s+s1}{\PYZsq{}}\PY{p}{)}
        \PY{n}{fig} \PY{o}{=} \PY{n}{plt}\PY{o}{.}\PY{n}{figure}\PY{p}{(}\PY{n}{figsize}\PY{o}{=}\PY{p}{(}\PY{l+m+mi}{7}\PY{p}{,}\PY{l+m+mi}{7}\PY{p}{)}\PY{p}{)}
        \PY{n}{plt}\PY{o}{.}\PY{n}{scatter}\PY{p}{(}\PY{n}{data}\PY{p}{[}\PY{p}{:}\PY{p}{,}\PY{l+m+mi}{0}\PY{p}{]}\PY{p}{,} \PY{n}{data}\PY{p}{[}\PY{p}{:}\PY{p}{,}\PY{l+m+mi}{1}\PY{p}{]}\PY{p}{,} \PY{n}{color}\PY{o}{=}\PY{l+s+s1}{\PYZsq{}}\PY{l+s+s1}{black}\PY{l+s+s1}{\PYZsq{}}\PY{p}{,} \PY{n}{alpha} \PY{o}{=} \PY{l+m+mf}{0.8}\PY{p}{)}
        \PY{n}{pltshow}\PY{p}{(}\PY{p}{)}
    
\PY{c+c1}{\PYZsh{}     Инициализация центроидов}
\PY{c+c1}{\PYZsh{}     Kmeans++ здесь, это первый центроид случайным образом остальные максимально удаленные от предыдущих}
    \PY{k}{if} \PY{p}{(}\PY{n}{s} \PY{o}{==} \PY{l+s+s1}{\PYZsq{}}\PY{l+s+s1}{KMeans++}\PY{l+s+s1}{\PYZsq{}}\PY{p}{)}\PY{p}{:}
        \PY{n}{centroids} \PY{o}{=} \PY{p}{\PYZob{}}\PY{l+m+mi}{1}\PY{p}{:} \PY{n}{data}\PY{p}{[}\PY{n}{np}\PY{o}{.}\PY{n}{random}\PY{o}{.}\PY{n}{randint}\PY{p}{(}\PY{l+m+mi}{0}\PY{p}{,} \PY{n}{n} \PY{o}{\PYZhy{}} \PY{l+m+mi}{1}\PY{p}{)}\PY{p}{]}\PY{o}{.}\PY{n}{tolist}\PY{p}{(}\PY{p}{)}\PY{p}{\PYZcb{}}
        \PY{k}{for} \PY{n}{it} \PY{o+ow}{in} \PY{n+nb}{range}\PY{p}{(}\PY{n}{k} \PY{o}{\PYZhy{}} \PY{l+m+mi}{1}\PY{p}{)}\PY{p}{:}
            \PY{n}{distss}\PY{o}{=}\PY{n}{np}\PY{o}{.}\PY{n}{full}\PY{p}{(}\PY{n}{n}\PY{p}{,} \PY{l+m+mi}{0}\PY{p}{)}
            \PY{k}{for} \PY{n}{i} \PY{o+ow}{in} \PY{n+nb}{range}\PY{p}{(}\PY{n}{n}\PY{p}{)}\PY{p}{:}
                \PY{n}{dist} \PY{o}{=} \PY{n}{np}\PY{o}{.}\PY{n}{inf}
                \PY{k}{for} \PY{n}{j} \PY{o+ow}{in} \PY{p}{(}\PY{n}{centroids}\PY{o}{.}\PY{n}{keys}\PY{p}{(}\PY{p}{)}\PY{p}{)}\PY{p}{:}
                    \PY{n}{new\PYZus{}dist} \PY{o}{=} \PY{l+m+mi}{0}
                    \PY{k}{for} \PY{n}{jj} \PY{o+ow}{in} \PY{n+nb}{range}\PY{p}{(}\PY{n}{m}\PY{p}{)}\PY{p}{:}
                        \PY{n}{new\PYZus{}dist} \PY{o}{+}\PY{o}{=} \PY{p}{(}\PY{n}{centroids}\PY{p}{[}\PY{n}{j}\PY{p}{]}\PY{p}{[}\PY{n}{jj}\PY{p}{]} \PY{o}{\PYZhy{}} \PY{n}{data}\PY{p}{[}\PY{n}{i}\PY{p}{]}\PY{p}{[}\PY{n}{jj}\PY{p}{]}\PY{p}{)} \PY{o}{*}\PY{o}{*} \PY{l+m+mi}{2}
                    \PY{n}{dist} \PY{o}{=} \PY{n+nb}{min}\PY{p}{(}\PY{n}{np}\PY{o}{.}\PY{n}{sqrt}\PY{p}{(}\PY{n}{new\PYZus{}dist}\PY{p}{)}\PY{p}{,} \PY{n}{dist}\PY{p}{)}
                \PY{n}{distss}\PY{p}{[}\PY{n}{i}\PY{p}{]} \PY{o}{=} \PY{n}{dist}
            \PY{n}{mx} \PY{o}{=} \PY{n}{np}\PY{o}{.}\PY{n}{max}\PY{p}{(}\PY{n}{distss}\PY{p}{)}
            \PY{n}{centroids}\PY{p}{[}\PY{n}{it} \PY{o}{+} \PY{l+m+mi}{2}\PY{p}{]} \PY{o}{=} \PY{n}{data}\PY{p}{[}\PY{n}{np}\PY{o}{.}\PY{n}{where}\PY{p}{(}\PY{n}{distss} \PY{o}{==} \PY{n}{mx}\PY{p}{)}\PY{p}{[}\PY{l+m+mi}{0}\PY{p}{]}\PY{p}{]}\PY{p}{[}\PY{l+m+mi}{0}\PY{p}{]}\PY{o}{.}\PY{n}{tolist}\PY{p}{(}\PY{p}{)}
    \PY{k}{else}\PY{p}{:}
\PY{c+c1}{\PYZsh{}         Инициализация случайным образом всех центроидов}
        \PY{n}{centroids} \PY{o}{=} \PY{p}{\PYZob{}} \PY{n}{i}\PY{o}{+}\PY{l+m+mi}{1}\PY{p}{:} \PY{n}{data}\PY{p}{[}\PY{n}{np}\PY{o}{.}\PY{n}{random}\PY{o}{.}\PY{n}{randint}\PY{p}{(}\PY{l+m+mi}{0}\PY{p}{,} \PY{n}{n} \PY{o}{\PYZhy{}} \PY{l+m+mi}{1}\PY{p}{)}\PY{p}{]}\PY{o}{.}\PY{n}{tolist}\PY{p}{(}\PY{p}{)} \PY{k}{for} \PY{n}{i} \PY{o+ow}{in} \PY{n+nb}{range}\PY{p}{(}\PY{n}{k}\PY{p}{)} \PY{p}{\PYZcb{}}
    
\PY{c+c1}{\PYZsh{}     Визуализация центроидов}
    \PY{k}{if} \PY{p}{(}\PY{n}{y}\PY{p}{)}\PY{p}{:}
        \PY{n+nb}{print}\PY{p}{(}\PY{l+s+s1}{\PYZsq{}}\PY{l+s+se}{\PYZbs{}n}\PY{l+s+se}{\PYZbs{}n}\PY{l+s+s1}{Визуализация центроидов}\PY{l+s+s1}{\PYZsq{}}\PY{p}{)}
        \PY{n}{cent}\PY{p}{(}\PY{p}{)}
        \PY{n}{pltshow}\PY{p}{(}\PY{p}{)}
    
\PY{c+c1}{\PYZsh{}     Функция для определния кластера для точки }
    \PY{n}{distances} \PY{o}{=} \PY{n}{np}\PY{o}{.}\PY{n}{full}\PY{p}{(}\PY{p}{(}\PY{n}{n}\PY{p}{,} \PY{n}{k}\PY{p}{)}\PY{p}{,} \PY{o}{\PYZhy{}}\PY{l+m+mi}{1}\PY{p}{)}
    \PY{k}{def} \PY{n+nf}{closest}\PY{p}{(}\PY{n}{data}\PY{p}{,} \PY{n}{centroids}\PY{p}{)}\PY{p}{:}
        \PY{k}{for} \PY{n}{i} \PY{o+ow}{in} \PY{n+nb}{range}\PY{p}{(}\PY{n}{n}\PY{p}{)}\PY{p}{:}
            \PY{k}{for} \PY{n}{j} \PY{o+ow}{in} \PY{n+nb}{range}\PY{p}{(}\PY{n}{k}\PY{p}{)}\PY{p}{:}
                \PY{n+nb}{sum} \PY{o}{=} \PY{l+m+mi}{0}
                \PY{k}{for} \PY{n}{l} \PY{o+ow}{in} \PY{n+nb}{range}\PY{p}{(}\PY{n}{m}\PY{p}{)}\PY{p}{:}
                    \PY{n+nb}{sum} \PY{o}{+}\PY{o}{=} \PY{p}{(}\PY{n}{data}\PY{p}{[}\PY{n}{i}\PY{p}{]}\PY{p}{[}\PY{n}{l}\PY{p}{]} \PY{o}{\PYZhy{}} \PY{n}{centroids}\PY{p}{[}\PY{n}{j} \PY{o}{+} \PY{l+m+mi}{1}\PY{p}{]}\PY{p}{[}\PY{n}{l}\PY{p}{]}\PY{p}{)}\PY{o}{*}\PY{o}{*}\PY{l+m+mi}{2}
                \PY{n}{distances}\PY{p}{[}\PY{n}{i}\PY{p}{]}\PY{p}{[}\PY{n}{j}\PY{p}{]} \PY{o}{=} \PY{n}{np}\PY{o}{.}\PY{n}{sqrt}\PY{p}{(}\PY{n+nb}{sum}\PY{p}{)}
        \PY{n}{closest\PYZus{}center} \PY{o}{=} \PY{n}{np}\PY{o}{.}\PY{n}{full}\PY{p}{(}\PY{n}{n}\PY{p}{,} \PY{o}{\PYZhy{}}\PY{l+m+mi}{1}\PY{p}{)}
        \PY{k}{for} \PY{n}{i} \PY{o+ow}{in} \PY{n+nb}{range}\PY{p}{(}\PY{n}{n}\PY{p}{)}\PY{p}{:}
            \PY{n}{closest\PYZus{}center}\PY{p}{[}\PY{n}{i}\PY{p}{]} \PY{o}{=} \PY{n}{np}\PY{o}{.}\PY{n}{argmin}\PY{p}{(}\PY{n}{distances}\PY{p}{[}\PY{n}{i}\PY{p}{]}\PY{p}{)} \PY{o}{+} \PY{l+m+mi}{1}
            \PY{n}{cl}\PY{p}{[}\PY{n}{i}\PY{p}{]} \PY{o}{=} \PY{n}{colmap}\PY{p}{[}\PY{p}{(}\PY{n}{np}\PY{o}{.}\PY{n}{argmin}\PY{p}{(}\PY{n}{distances}\PY{p}{[}\PY{n}{i}\PY{p}{]}\PY{p}{)} \PY{o}{+} \PY{l+m+mi}{1} \PY{p}{)}\PY{o}{\PYZpc{}} \PY{l+m+mi}{6} \PY{o}{+} \PY{l+m+mi}{1}\PY{p}{]}
        \PY{k}{return} \PY{n}{closest\PYZus{}center}

\PY{c+c1}{\PYZsh{}    Первое присваивания объектов центроидам}
    \PY{n}{distribution} \PY{o}{=} \PY{n}{closest}\PY{p}{(}\PY{n}{data}\PY{p}{,} \PY{n}{centroids}\PY{p}{)}


\PY{c+c1}{\PYZsh{}     Визуализация к какому классу относятся объкты после инициализации центроидов}
    \PY{k}{if} \PY{p}{(}\PY{n}{y}\PY{p}{)}\PY{p}{:}
        \PY{n+nb}{print}\PY{p}{(}\PY{l+s+s1}{\PYZsq{}}\PY{l+s+se}{\PYZbs{}n}\PY{l+s+se}{\PYZbs{}n}\PY{l+s+s1}{Визуализация к какому классу относятся объкты после инициализации центроидов}\PY{l+s+s1}{\PYZsq{}}\PY{p}{)}
        \PY{n}{cent}\PY{p}{(}\PY{p}{)}
        \PY{n}{poin}\PY{p}{(}\PY{p}{)}
        \PY{n}{pltshow}\PY{p}{(}\PY{p}{)}
    
\PY{c+c1}{\PYZsh{}     Глубокая копия создает новый составной объект, и затем рекурсивно вставляет в него копии объектов, находящихся в оригинале. }
    \PY{n}{old\PYZus{}centroids} \PY{o}{=} \PY{n}{copy}\PY{o}{.}\PY{n}{deepcopy}\PY{p}{(}\PY{n}{centroids}\PY{p}{)}

\PY{c+c1}{\PYZsh{}     Функция обновления центроидов}
    \PY{k}{def} \PY{n+nf}{update}\PY{p}{(}\PY{n}{distr}\PY{p}{,} \PY{n}{data}\PY{p}{)}\PY{p}{:}
        \PY{k}{for} \PY{n}{i} \PY{o+ow}{in} \PY{n}{centroids}\PY{o}{.}\PY{n}{keys}\PY{p}{(}\PY{p}{)}\PY{p}{:}
            \PY{n}{centroids}\PY{p}{[}\PY{n}{i}\PY{p}{]}\PY{p}{[}\PY{l+m+mi}{0}\PY{p}{]} \PY{o}{=} \PY{n}{np}\PY{o}{.}\PY{n}{mean}\PY{p}{(}\PY{n}{data}\PY{p}{[}\PY{n}{np}\PY{o}{.}\PY{n}{where}\PY{p}{(}\PY{n}{distribution} \PY{o}{==} \PY{n}{i}\PY{p}{)}\PY{p}{[}\PY{l+m+mi}{0}\PY{p}{]}\PY{p}{]}\PY{p}{[}\PY{p}{:}\PY{p}{,}\PY{l+m+mi}{0}\PY{p}{]}\PY{p}{)}
            \PY{n}{centroids}\PY{p}{[}\PY{n}{i}\PY{p}{]}\PY{p}{[}\PY{l+m+mi}{1}\PY{p}{]} \PY{o}{=} \PY{n}{np}\PY{o}{.}\PY{n}{mean}\PY{p}{(}\PY{n}{data}\PY{p}{[}\PY{n}{np}\PY{o}{.}\PY{n}{where}\PY{p}{(}\PY{n}{distribution} \PY{o}{==} \PY{n}{i}\PY{p}{)}\PY{p}{[}\PY{l+m+mi}{0}\PY{p}{]}\PY{p}{]}\PY{p}{[}\PY{p}{:}\PY{p}{,}\PY{l+m+mi}{1}\PY{p}{]}\PY{p}{)}
        \PY{k}{return} \PY{n}{centroids}

\PY{c+c1}{\PYZsh{}     Обновление центроидов}
    \PY{n}{centroids} \PY{o}{=} \PY{n}{update}\PY{p}{(}\PY{n}{distribution}\PY{p}{,} \PY{n}{data}\PY{p}{)}
    
\PY{c+c1}{\PYZsh{}     Визуализация после одного обновления }
    \PY{k}{if} \PY{p}{(}\PY{n}{y}\PY{p}{)}\PY{p}{:}
        \PY{n+nb}{print}\PY{p}{(}\PY{l+s+s1}{\PYZsq{}}\PY{l+s+se}{\PYZbs{}n}\PY{l+s+se}{\PYZbs{}n}\PY{l+s+s1}{Визуализация после одного обновления положения центроидов}\PY{l+s+s1}{\PYZsq{}}\PY{p}{)}
        \PY{n}{cent}\PY{p}{(}\PY{p}{)}
        \PY{n}{poin}\PY{p}{(}\PY{p}{)}
        \PY{n}{pltshow}\PY{p}{(}\PY{p}{)}
    
\PY{c+c1}{\PYZsh{}     Обновление точек(так как центроиды сдвинулись)  }
   
    \PY{n}{distribution} \PY{o}{=} \PY{n}{closest}\PY{p}{(}\PY{n}{data}\PY{p}{,} \PY{n}{centroids}\PY{p}{)}

\PY{c+c1}{\PYZsh{}     Обновление центроидов пока изменения не перстанут происходить или 100 итераций}
    \PY{n}{n\PYZus{}iter} \PY{o}{=} \PY{l+m+mi}{100}
    \PY{k}{while} \PY{k+kc}{True} \PY{o+ow}{or} \PY{p}{(}\PY{n}{n\PYZus{}iter} \PY{o}{!=} \PY{l+m+mi}{0}\PY{p}{)}\PY{p}{:}
        \PY{n}{closest\PYZus{}centroids} \PY{o}{=} \PY{n}{distribution}
        \PY{n}{centroids} \PY{o}{=} \PY{n}{update}\PY{p}{(}\PY{n}{distribution}\PY{p}{,} \PY{n}{data}\PY{p}{)}
        \PY{n}{distribution} \PY{o}{=} \PY{n}{closest}\PY{p}{(}\PY{n}{data}\PY{p}{,} \PY{n}{centroids}\PY{p}{)}
        \PY{n}{n\PYZus{}iter} \PY{o}{\PYZhy{}}\PY{o}{=} \PY{l+m+mi}{1}
        \PY{k}{if} \PY{n}{np}\PY{o}{.}\PY{n}{array\PYZus{}equal}\PY{p}{(}\PY{n}{closest\PYZus{}centroids}\PY{p}{,} \PY{n}{distribution}\PY{p}{)}\PY{p}{:}
            \PY{k}{break}

\PY{c+c1}{\PYZsh{}     Визуализация окончательного результата}
    \PY{k}{if} \PY{p}{(}\PY{n}{y}\PY{p}{)}\PY{p}{:}
        \PY{n+nb}{print}\PY{p}{(}\PY{l+s+s1}{\PYZsq{}}\PY{l+s+se}{\PYZbs{}n}\PY{l+s+se}{\PYZbs{}n}\PY{l+s+s1}{Визуализация окончательного результата:}\PY{l+s+s1}{\PYZsq{}}\PY{p}{)}
        \PY{n}{cent}\PY{p}{(}\PY{p}{)}
        \PY{n}{poin}\PY{p}{(}\PY{p}{)}
        \PY{n}{pltshow}\PY{p}{(}\PY{p}{)}
    \PY{k}{return} \PY{n}{distribution}\PY{p}{,} \PY{n}{cl}
\end{Verbatim}
\end{tcolorbox}

    \begin{tcolorbox}[breakable, size=fbox, boxrule=1pt, pad at break*=1mm,colback=cellbackground, colframe=cellborder]
\prompt{In}{incolor}{186}{\hspace{4pt}}
\begin{Verbatim}[commandchars=\\\{\}]
\PY{c+c1}{\PYZsh{} Метод логтя для нахождения количества кластеров}
\PY{k}{def} \PY{n+nf}{elbow\PYZus{}method}\PY{p}{(}\PY{n}{data}\PY{p}{,}\PY{n}{a}\PY{p}{,}\PY{n}{b}\PY{p}{)}\PY{p}{:}
    \PY{n}{silh\PYZus{}score} \PY{o}{=} \PY{p}{[}\PY{p}{]}
    \PY{k}{for} \PY{n}{k} \PY{o+ow}{in} \PY{n+nb}{range}\PY{p}{(}\PY{n}{a}\PY{p}{,} \PY{n}{b}\PY{p}{)}\PY{p}{:}
        \PY{n}{model}\PY{p}{,} \PY{n}{cl\PYZus{}model} \PY{o}{=} \PY{n}{myKMeans}\PY{p}{(}\PY{n}{k}\PY{p}{,} \PY{n}{data}\PY{p}{,} \PY{l+m+mi}{0}\PY{p}{,}\PY{l+s+s1}{\PYZsq{}}\PY{l+s+s1}{KMeans++}\PY{l+s+s1}{\PYZsq{}}\PY{p}{)}
        \PY{n+nb}{print}\PY{p}{(}\PY{l+s+s1}{\PYZsq{}}\PY{l+s+s1}{Running comput. for}\PY{l+s+s1}{\PYZsq{}}\PY{p}{,} \PY{n}{k}\PY{p}{,}\PY{l+s+s1}{\PYZsq{}}\PY{l+s+s1}{cluster(s)}\PY{l+s+s1}{\PYZsq{}}\PY{p}{)}
        \PY{n}{silh\PYZus{}score}\PY{o}{.}\PY{n}{append}\PY{p}{(}\PY{n}{silhouette\PYZus{}score}\PY{p}{(}\PY{n}{data}\PY{p}{,} \PY{n}{model}\PY{p}{)}\PY{p}{)}
    \PY{n}{plt}\PY{o}{.}\PY{n}{plot}\PY{p}{(}\PY{n+nb}{range}\PY{p}{(}\PY{n}{a}\PY{p}{,} \PY{n}{b}\PY{p}{)}\PY{p}{,} \PY{n}{silh\PYZus{}score}\PY{p}{)}
\end{Verbatim}
\end{tcolorbox}

    \begin{tcolorbox}[breakable, size=fbox, boxrule=1pt, pad at break*=1mm,colback=cellbackground, colframe=cellborder]
\prompt{In}{incolor}{155}{\hspace{4pt}}
\begin{Verbatim}[commandchars=\\\{\}]
\PY{c+c1}{\PYZsh{} Функция упорядочевания data:}
\PY{k}{def} \PY{n+nf}{order\PYZus{}data}\PY{p}{(}\PY{n}{data}\PY{p}{,} \PY{n}{model}\PY{p}{)}\PY{p}{:}
    \PY{n}{data\PYZus{}clustered} \PY{o}{=} \PY{n}{np}\PY{o}{.}\PY{n}{array}\PY{p}{(}\PY{p}{[}\PY{l+m+mi}{0}\PY{p}{,}\PY{l+m+mi}{0}\PY{p}{]}\PY{p}{)}
    \PY{k}{for} \PY{n}{i} \PY{o+ow}{in} \PY{n+nb}{range}\PY{p}{(}\PY{l+m+mi}{1}\PY{p}{,}\PY{n+nb}{int}\PY{p}{(}\PY{n}{np}\PY{o}{.}\PY{n}{max}\PY{p}{(}\PY{n}{model}\PY{p}{)}\PY{p}{)}\PY{o}{+}\PY{l+m+mi}{1}\PY{p}{)}\PY{p}{:}
        \PY{k}{for} \PY{n}{j} \PY{o+ow}{in} \PY{n+nb}{range}\PY{p}{(}\PY{n+nb}{len}\PY{p}{(}\PY{n}{model}\PY{p}{)}\PY{p}{)}\PY{p}{:}
            \PY{k}{if}\PY{p}{(}\PY{n}{model}\PY{p}{[}\PY{n}{j}\PY{p}{]} \PY{o}{==} \PY{n}{i}\PY{p}{)}\PY{p}{:}
                \PY{n}{data\PYZus{}clustered} \PY{o}{=} \PY{n}{np}\PY{o}{.}\PY{n}{vstack}\PY{p}{(}\PY{p}{(}\PY{n}{data\PYZus{}clustered}\PY{p}{,} \PY{n}{data}\PY{p}{[}\PY{n}{j}\PY{p}{]}\PY{p}{)}\PY{p}{)}
    \PY{n}{data\PYZus{}clustered} \PY{o}{=} \PY{n}{np}\PY{o}{.}\PY{n}{delete}\PY{p}{(}\PY{n}{data\PYZus{}clustered}\PY{p}{,} \PY{l+m+mi}{0}\PY{p}{,} \PY{n}{axis} \PY{o}{=} \PY{l+m+mi}{0}\PY{p}{)} 
    \PY{k}{return} \PY{n}{data\PYZus{}clustered}

\PY{c+c1}{\PYZsh{} Функция для визуализации матрицы попарных состояний:}
\PY{k}{def} \PY{n+nf}{visualization}\PY{p}{(}\PY{n}{data}\PY{p}{,} \PY{n}{model}\PY{p}{,} \PY{n}{cl\PYZus{}model}\PY{p}{)}\PY{p}{:}
    \PY{n+nb}{print}\PY{p}{(}\PY{l+s+s1}{\PYZsq{}}\PY{l+s+s1}{Матрица попарных расстояний до упорядочевания:}\PY{l+s+s1}{\PYZsq{}}\PY{p}{)}
    \PY{n}{fig} \PY{o}{=} \PY{n}{plt}\PY{o}{.}\PY{n}{figure}\PY{p}{(}\PY{n}{figsize}\PY{o}{=}\PY{p}{(}\PY{l+m+mi}{7}\PY{p}{,}\PY{l+m+mi}{7}\PY{p}{)}\PY{p}{)}
    \PY{n}{plt}\PY{o}{.}\PY{n}{scatter}\PY{p}{(}\PY{n}{data}\PY{p}{[}\PY{p}{:}\PY{p}{,}\PY{l+m+mi}{0}\PY{p}{]}\PY{p}{,}\PY{n}{data}\PY{p}{[}\PY{p}{:}\PY{p}{,}\PY{l+m+mi}{1}\PY{p}{]}\PY{p}{)}
    \PY{n}{plt}\PY{o}{.}\PY{n}{show}\PY{p}{(}\PY{p}{)}
    \PY{n}{plt}\PY{o}{.}\PY{n}{matshow}\PY{p}{(}\PY{n}{pairwise\PYZus{}distances}\PY{p}{(}\PY{n}{data}\PY{p}{)}\PY{p}{,} \PY{n}{aspect}\PY{o}{=}\PY{l+s+s1}{\PYZsq{}}\PY{l+s+s1}{auto}\PY{l+s+s1}{\PYZsq{}}\PY{p}{)}
    \PY{n}{plt}\PY{o}{.}\PY{n}{colorbar}\PY{p}{(}\PY{p}{)}
    \PY{n}{plt}\PY{o}{.}\PY{n}{show}\PY{p}{(}\PY{p}{)}
    \PY{n}{fig} \PY{o}{=} \PY{n}{plt}\PY{o}{.}\PY{n}{figure}\PY{p}{(}\PY{n}{figsize}\PY{o}{=}\PY{p}{(}\PY{l+m+mi}{7}\PY{p}{,}\PY{l+m+mi}{7}\PY{p}{)}\PY{p}{)}
    \PY{n}{plt}\PY{o}{.}\PY{n}{scatter}\PY{p}{(}\PY{n}{data}\PY{p}{[}\PY{p}{:}\PY{p}{,}\PY{l+m+mi}{0}\PY{p}{]}\PY{p}{,}\PY{n}{data}\PY{p}{[}\PY{p}{:}\PY{p}{,}\PY{l+m+mi}{1}\PY{p}{]}\PY{p}{,} \PY{n}{c} \PY{o}{=} \PY{n}{cl\PYZus{}model}\PY{p}{)}
    \PY{n}{plt}\PY{o}{.}\PY{n}{show}\PY{p}{(}\PY{p}{)}
    \PY{n}{plt}\PY{o}{.}\PY{n}{matshow}\PY{p}{(}\PY{n}{pairwise\PYZus{}distances}\PY{p}{(}\PY{n}{order\PYZus{}data}\PY{p}{(}\PY{n}{data}\PY{p}{,} \PY{n}{model}\PY{p}{)}\PY{p}{)}\PY{p}{,} \PY{n}{aspect}\PY{o}{=}\PY{l+s+s1}{\PYZsq{}}\PY{l+s+s1}{auto}\PY{l+s+s1}{\PYZsq{}}\PY{p}{)}
    \PY{n}{plt}\PY{o}{.}\PY{n}{colorbar}\PY{p}{(}\PY{p}{)}
    \PY{n}{plt}\PY{o}{.}\PY{n}{show}\PY{p}{(}\PY{p}{)}
\end{Verbatim}
\end{tcolorbox}

    \hypertarget{ux43fux435ux440ux432ux44bux439-dataset-s1-ux441ux43aux430ux447ux430ux43d-ux43eux442ux441ux44eux434ux430-httpcs.joensuu.fisipudatasets}{%
\subsubsection{Первый dataset ``s1'' скачан отсюда:
http://cs.joensuu.fi/sipu/datasets/}\label{ux43fux435ux440ux432ux44bux439-dataset-s1-ux441ux43aux430ux447ux430ux43d-ux43eux442ux441ux44eux434ux430-httpcs.joensuu.fisipudatasets}}

    \begin{tcolorbox}[breakable, size=fbox, boxrule=1pt, pad at break*=1mm,colback=cellbackground, colframe=cellborder]
\prompt{In}{incolor}{187}{\hspace{4pt}}
\begin{Verbatim}[commandchars=\\\{\}]
\PY{n}{data} \PY{o}{=} \PY{n}{np}\PY{o}{.}\PY{n}{loadtxt}\PY{p}{(}\PY{l+s+s2}{\PYZdq{}}\PY{l+s+s2}{/Users/ofirserovlad/Desktop/Kaggle/K\PYZhy{}means/s1.txt}\PY{l+s+s2}{\PYZdq{}}\PY{p}{,} \PY{n}{dtype}\PY{o}{=}\PY{l+s+s2}{\PYZdq{}}\PY{l+s+s2}{int}\PY{l+s+s2}{\PYZdq{}}\PY{p}{)}
\end{Verbatim}
\end{tcolorbox}

    \hypertarget{ux43eux434ux43dux438ux43c-ux438ux437-ux43cux438ux43dux443ux441ux43eux432-ux434ux430ux43dux43dux43eux433ux43e-ux43cux435ux442ux43eux434ux430-ux44fux432ux43bux44fux435ux442ux441ux44f-ux442ux43e-ux447ux442ux43e-ux43cux44b-ux434ux43eux43bux436ux43dux44b-ux437ux43dux430ux442ux44c-ux43aux43eux43bux438ux447ux435ux441ux442ux432ux43e-ux43aux43bux430ux441ux442ux435ux440ux43eux432-ux437ux430ux440ux430ux43dux435ux435}{%
\subsubsection{Одним из минусов данного метода является то, что мы
должны знать количество кластеров
заранее}\label{ux43eux434ux43dux438ux43c-ux438ux437-ux43cux438ux43dux443ux441ux43eux432-ux434ux430ux43dux43dux43eux433ux43e-ux43cux435ux442ux43eux434ux430-ux44fux432ux43bux44fux435ux442ux441ux44f-ux442ux43e-ux447ux442ux43e-ux43cux44b-ux434ux43eux43bux436ux43dux44b-ux437ux43dux430ux442ux44c-ux43aux43eux43bux438ux447ux435ux441ux442ux432ux43e-ux43aux43bux430ux441ux442ux435ux440ux43eux432-ux437ux430ux440ux430ux43dux435ux435}}

\hypertarget{ux434ux43bux44f-ux43dux430ux445ux43eux436ux434ux435ux43dux438ux44f-ux447ux438ux441ux43bux430-ux43aux43bux430ux441ux442ux435ux440ux43eux432-ux432ux43eux441ux43fux43eux43bux44cux437ux443ux435ux43cux441ux44f-ux43cux435ux442ux43eux434ux43eux43c-ux43bux43eux433ux442ux44f.-ux43cux435ux442ux440ux438ux43aux430-ux434ux430ux43dux43dux43eux433ux43e-ux43cux435ux442ux43eux434ux430-ux43fux43eux43aux430ux437ux44bux432ux430ux435ux442-ux43eux442ux43dux43eux448ux435ux43dux438ux435-ux432ux43dux443ux442ux440ux438ux43aux43bux430ux441ux442ux435ux440ux43dux43eux433ux43e-ux440ux430ux441ux441ux442ux43eux44fux43dux438ux44f-ux43a-ux43cux435ux436ux43aux43bux430ux441ux442ux435ux440ux43dux43eux43cux443-ux440ux430ux441ux441ux442ux43eux44fux43dux438ux44e.-ux442ux430ux43aux438ux43c-ux43eux431ux440ux430ux437ux43eux43c-ux43cux44b-ux43cux43eux436ux435ux43c-ux443ux432ux438ux434ux435ux442ux44c-ux43dux443ux436ux43dux43eux435-ux43aux43eux43bux438ux447ux435ux441ux442ux432ux43e-ux43aux43bux430ux441ux442ux435ux440ux43eux432.}{%
\subsubsection{Для нахождения числа кластеров воспользуемся ``Методом
логтя''. Метрика данного метода показывает отношение внутрикластерного
расстояния к межкластерному расстоянию. Таким образом мы можем
``увидеть'' нужное количество
кластеров.}\label{ux434ux43bux44f-ux43dux430ux445ux43eux436ux434ux435ux43dux438ux44f-ux447ux438ux441ux43bux430-ux43aux43bux430ux441ux442ux435ux440ux43eux432-ux432ux43eux441ux43fux43eux43bux44cux437ux443ux435ux43cux441ux44f-ux43cux435ux442ux43eux434ux43eux43c-ux43bux43eux433ux442ux44f.-ux43cux435ux442ux440ux438ux43aux430-ux434ux430ux43dux43dux43eux433ux43e-ux43cux435ux442ux43eux434ux430-ux43fux43eux43aux430ux437ux44bux432ux430ux435ux442-ux43eux442ux43dux43eux448ux435ux43dux438ux435-ux432ux43dux443ux442ux440ux438ux43aux43bux430ux441ux442ux435ux440ux43dux43eux433ux43e-ux440ux430ux441ux441ux442ux43eux44fux43dux438ux44f-ux43a-ux43cux435ux436ux43aux43bux430ux441ux442ux435ux440ux43dux43eux43cux443-ux440ux430ux441ux441ux442ux43eux44fux43dux438ux44e.-ux442ux430ux43aux438ux43c-ux43eux431ux440ux430ux437ux43eux43c-ux43cux44b-ux43cux43eux436ux435ux43c-ux443ux432ux438ux434ux435ux442ux44c-ux43dux443ux436ux43dux43eux435-ux43aux43eux43bux438ux447ux435ux441ux442ux432ux43e-ux43aux43bux430ux441ux442ux435ux440ux43eux432.}}

    \begin{tcolorbox}[breakable, size=fbox, boxrule=1pt, pad at break*=1mm,colback=cellbackground, colframe=cellborder]
\prompt{In}{incolor}{189}{\hspace{4pt}}
\begin{Verbatim}[commandchars=\\\{\}]
\PY{n}{elbow\PYZus{}method}\PY{p}{(}\PY{n}{data}\PY{p}{,} \PY{l+m+mi}{3}\PY{p}{,} \PY{l+m+mi}{25}\PY{p}{)}
\end{Verbatim}
\end{tcolorbox}

    \begin{Verbatim}[commandchars=\\\{\}]
Running comput. for 3 cluster(s)
Running comput. for 4 cluster(s)
Running comput. for 5 cluster(s)
Running comput. for 6 cluster(s)
Running comput. for 7 cluster(s)
Running comput. for 8 cluster(s)
Running comput. for 9 cluster(s)
Running comput. for 10 cluster(s)
Running comput. for 11 cluster(s)
Running comput. for 12 cluster(s)
Running comput. for 13 cluster(s)
Running comput. for 14 cluster(s)
Running comput. for 15 cluster(s)
Running comput. for 16 cluster(s)
Running comput. for 17 cluster(s)
Running comput. for 18 cluster(s)
Running comput. for 19 cluster(s)
Running comput. for 20 cluster(s)
Running comput. for 21 cluster(s)
Running comput. for 22 cluster(s)
Running comput. for 23 cluster(s)
Running comput. for 24 cluster(s)
\end{Verbatim}

    \begin{center}
    \adjustimage{max size={0.9\linewidth}{0.9\paperheight}}{output_8_1.png}
    \end{center}
    { \hspace*{\fill} \\}
    
    \hypertarget{ux446ux435ux43dux442ux440ux43eux438ux434ux44b-ux43eux431ux43eux437ux43dux430ux447ux435ux43dux44b-ux441ux438ux43cux432ux43eux43bux43eux43c-ux432-ux441ux43bux443ux447ux430ux435-ux43fux43bux43eux445ux43eux439-ux432ux438ux434ux438ux43cux43eux441ux442ux438-ux446ux435ux43dux442ux440ux43eux438ux434ux43eux432-ux441ux442ux43eux438ux442-ux443ux43cux435ux43dux44cux448ux438ux442ux44c-ux43fux440ux43eux437ux440ux430ux447ux43dux43eux441ux442ux44c-ux433ux438ux43fux435ux440ux43fux430ux440ux430ux43cux435ux442ux440-alpha-ux432-16-ux441ux442ux440ux43eux43aux435-ux444ux443ux43dux43aux446ux438ux438-mykmeans-ux432-ux432ux438ux437ux443ux430ux43bux438ux437ux430ux446ux438ux438-ux438ux441ux43fux43eux43bux44cux437ux443ux44eux442ux441ux44f-ux442ux43eux43bux44cux43aux43e-6-ux446ux432ux435ux442ux43eux432-ux431ux43eux43bux44cux448ux435ux433ux43e-ux43aux43eux43bux438ux447ux435ux441ux442ux432ux430-ux44fux440ux43aux438ux445-ux446ux432ux435ux442ux43eux432-ux438ux437-matplotlib-ux432ux44bux43dux435ux441ux442ux438-ux44f-ux43dux435-ux441ux43cux43eux433.-ux43dux435-ux441ux442ux43eux438ux442-ux434ux443ux43cux430ux442ux44c-ux447ux442ux43e-ux432ux441ux435-ux43eux431ux44aux435ux43aux442ux44b-ux43eux434ux43dux43eux433ux43e-ux446ux432ux435ux442ux430-ux43fux440ux438ux43dux430ux434ux43bux435ux436ux430ux442-ux43eux434ux43dux43eux43cux443-ux43aux43bux430ux441ux442ux435ux440ux443.-ux43dux430-ux43fux440ux438ux43cux435ux440ux435-ux43dux438ux436ux435-ux442ux43eux43bux44cux43aux43e-ux434ux432ux435-ux43dux438ux436ux43dux438ux445-ux433ux440ux443ux43fux43fux44b-ux442ux43eux447ux435ux43a-ux43eux431ux44aux435ux434ux435ux43dux435ux43dux44b-ux432-ux43eux434ux438ux43d-ux43aux43bux430ux441ux442ux435ux440-ux43dux435ux43fux440ux430ux432ux438ux43bux44cux43dux43e-ux438-ux43eux434ux43dux430-ux433ux440ux443ux43fux43fux430-ux442ux43eux447ux435ux43a-ux440ux430ux437ux431ux438ux442ux430-ux43dux430-ux434ux432ux430-ux43aux43bux430ux441ux442ux435ux440ux430-ux44dux442ux43e-ux441ux432ux44fux437ux430ux43dux43e-ux441-ux442ux435ux43c-ux447ux442ux43e-ux438ux437ux43dux430ux447ux430ux43bux44cux43dux43e-3-ux446ux435ux43dux442ux440ux43eux438ux434ux430-ux431ux44bux43bux438-ux438ux43dux438ux446ux438ux430ux43bux438ux440ux43eux432ux430ux43dux44b-ux43fux43eux447ux442ux438-ux432-ux43eux434ux43dux43eux439-ux442ux43eux447ux43aux435.-ux43cux43eux436ux435ux43c-ux443ux441ux442ux440ux43eux43dux438ux442ux44c-ux442ux430ux43aux438ux435-ux43fux440ux43eux431ux43bux435ux43cux44b-ux43cux435ux442ux43eux434ux43eux43c-ux441-ux43fux43eux43cux43eux449ux44cux44e-ux43aux43eux442ux43eux440ux43eux433ux43e-ux446ux435ux442ux440ux43eux438ux434ux44b-ux438ux43dux438ux446ux438ux430ux43bux438ux437ux438ux440ux443ux44eux442ux441ux44f-ux43dux430-ux43cux430ux43aux441ux438ux43cux430ux43bux44cux43dux43eux43c-ux440ux430ux441ux441ux442ux43eux44fux43dux438ux438-ux434ux440ux443ux433-ux43eux442-ux434ux440ux443ux433ux430}{%
\subsubsection{Центроиды обозначены символом * \#\#\#\#\# (в случае
плохой видимости центроидов, стоит уменьшить прозрачность, гиперпараметр
alpha в 16 строке функции myKMeans) \#\#\# В визуализации используются
только 6 цветов, большего количества ярких цветов из matplotlib вынести
я не смог. Не стоит думать, что все объекты одного цвета принадлежат
одному кластеру. \#\#\# На примере ниже только две нижних группы точек
объеденены в один кластер неправильно и одна группа точек разбита на два
кластера, это связано с тем, что изначально 3 центроида были
инициалированы почти в одной точке. Можем устронить такие проблемы
методом, с помощью которого цетроиды инициализируются на максимальном
расстоянии друг от
друга}\label{ux446ux435ux43dux442ux440ux43eux438ux434ux44b-ux43eux431ux43eux437ux43dux430ux447ux435ux43dux44b-ux441ux438ux43cux432ux43eux43bux43eux43c-ux432-ux441ux43bux443ux447ux430ux435-ux43fux43bux43eux445ux43eux439-ux432ux438ux434ux438ux43cux43eux441ux442ux438-ux446ux435ux43dux442ux440ux43eux438ux434ux43eux432-ux441ux442ux43eux438ux442-ux443ux43cux435ux43dux44cux448ux438ux442ux44c-ux43fux440ux43eux437ux440ux430ux447ux43dux43eux441ux442ux44c-ux433ux438ux43fux435ux440ux43fux430ux440ux430ux43cux435ux442ux440-alpha-ux432-16-ux441ux442ux440ux43eux43aux435-ux444ux443ux43dux43aux446ux438ux438-mykmeans-ux432-ux432ux438ux437ux443ux430ux43bux438ux437ux430ux446ux438ux438-ux438ux441ux43fux43eux43bux44cux437ux443ux44eux442ux441ux44f-ux442ux43eux43bux44cux43aux43e-6-ux446ux432ux435ux442ux43eux432-ux431ux43eux43bux44cux448ux435ux433ux43e-ux43aux43eux43bux438ux447ux435ux441ux442ux432ux430-ux44fux440ux43aux438ux445-ux446ux432ux435ux442ux43eux432-ux438ux437-matplotlib-ux432ux44bux43dux435ux441ux442ux438-ux44f-ux43dux435-ux441ux43cux43eux433.-ux43dux435-ux441ux442ux43eux438ux442-ux434ux443ux43cux430ux442ux44c-ux447ux442ux43e-ux432ux441ux435-ux43eux431ux44aux435ux43aux442ux44b-ux43eux434ux43dux43eux433ux43e-ux446ux432ux435ux442ux430-ux43fux440ux438ux43dux430ux434ux43bux435ux436ux430ux442-ux43eux434ux43dux43eux43cux443-ux43aux43bux430ux441ux442ux435ux440ux443.-ux43dux430-ux43fux440ux438ux43cux435ux440ux435-ux43dux438ux436ux435-ux442ux43eux43bux44cux43aux43e-ux434ux432ux435-ux43dux438ux436ux43dux438ux445-ux433ux440ux443ux43fux43fux44b-ux442ux43eux447ux435ux43a-ux43eux431ux44aux435ux434ux435ux43dux435ux43dux44b-ux432-ux43eux434ux438ux43d-ux43aux43bux430ux441ux442ux435ux440-ux43dux435ux43fux440ux430ux432ux438ux43bux44cux43dux43e-ux438-ux43eux434ux43dux430-ux433ux440ux443ux43fux43fux430-ux442ux43eux447ux435ux43a-ux440ux430ux437ux431ux438ux442ux430-ux43dux430-ux434ux432ux430-ux43aux43bux430ux441ux442ux435ux440ux430-ux44dux442ux43e-ux441ux432ux44fux437ux430ux43dux43e-ux441-ux442ux435ux43c-ux447ux442ux43e-ux438ux437ux43dux430ux447ux430ux43bux44cux43dux43e-3-ux446ux435ux43dux442ux440ux43eux438ux434ux430-ux431ux44bux43bux438-ux438ux43dux438ux446ux438ux430ux43bux438ux440ux43eux432ux430ux43dux44b-ux43fux43eux447ux442ux438-ux432-ux43eux434ux43dux43eux439-ux442ux43eux447ux43aux435.-ux43cux43eux436ux435ux43c-ux443ux441ux442ux440ux43eux43dux438ux442ux44c-ux442ux430ux43aux438ux435-ux43fux440ux43eux431ux43bux435ux43cux44b-ux43cux435ux442ux43eux434ux43eux43c-ux441-ux43fux43eux43cux43eux449ux44cux44e-ux43aux43eux442ux43eux440ux43eux433ux43e-ux446ux435ux442ux440ux43eux438ux434ux44b-ux438ux43dux438ux446ux438ux430ux43bux438ux437ux438ux440ux443ux44eux442ux441ux44f-ux43dux430-ux43cux430ux43aux441ux438ux43cux430ux43bux44cux43dux43eux43c-ux440ux430ux441ux441ux442ux43eux44fux43dux438ux438-ux434ux440ux443ux433-ux43eux442-ux434ux440ux443ux433ux430}}

    \begin{tcolorbox}[breakable, size=fbox, boxrule=1pt, pad at break*=1mm,colback=cellbackground, colframe=cellborder]
\prompt{In}{incolor}{190}{\hspace{4pt}}
\begin{Verbatim}[commandchars=\\\{\}]
\PY{o}{\PYZpc{}\PYZpc{}time}
\PY{n}{model}\PY{p}{,} \PY{n}{cl\PYZus{}model} \PY{o}{=} \PY{n}{myKMeans}\PY{p}{(}\PY{l+m+mi}{15}\PY{p}{,} \PY{n}{data}\PY{p}{,} \PY{l+m+mi}{1}\PY{p}{,} \PY{l+s+s1}{\PYZsq{}}\PY{l+s+s1}{KMeans}\PY{l+s+s1}{\PYZsq{}}\PY{p}{)}
\end{Verbatim}
\end{tcolorbox}

    \begin{Verbatim}[commandchars=\\\{\}]


NUMBER OF CLUSTERS: 15




Визуализация данных
\end{Verbatim}

    \begin{center}
    \adjustimage{max size={0.9\linewidth}{0.9\paperheight}}{output_10_1.png}
    \end{center}
    { \hspace*{\fill} \\}
    
    \begin{Verbatim}[commandchars=\\\{\}]


Визуализация центроидов
\end{Verbatim}

    \begin{center}
    \adjustimage{max size={0.9\linewidth}{0.9\paperheight}}{output_10_3.png}
    \end{center}
    { \hspace*{\fill} \\}
    
    \begin{Verbatim}[commandchars=\\\{\}]


Визуализация к какому классу относятся объкты после инициализации центроидов
\end{Verbatim}

    \begin{center}
    \adjustimage{max size={0.9\linewidth}{0.9\paperheight}}{output_10_5.png}
    \end{center}
    { \hspace*{\fill} \\}
    
    \begin{Verbatim}[commandchars=\\\{\}]


Визуализация после одного обновления положения центроидов
\end{Verbatim}

    \begin{center}
    \adjustimage{max size={0.9\linewidth}{0.9\paperheight}}{output_10_7.png}
    \end{center}
    { \hspace*{\fill} \\}
    
    \begin{Verbatim}[commandchars=\\\{\}]


Визуализация окончательного результата:
\end{Verbatim}

    \begin{center}
    \adjustimage{max size={0.9\linewidth}{0.9\paperheight}}{output_10_9.png}
    \end{center}
    { \hspace*{\fill} \\}
    
    \begin{Verbatim}[commandchars=\\\{\}]
CPU times: user 31.9 s, sys: 268 ms, total: 32.2 s
Wall time: 32.6 s
\end{Verbatim}

    \begin{tcolorbox}[breakable, size=fbox, boxrule=1pt, pad at break*=1mm,colback=cellbackground, colframe=cellborder]
\prompt{In}{incolor}{191}{\hspace{4pt}}
\begin{Verbatim}[commandchars=\\\{\}]
\PY{o}{\PYZpc{}\PYZpc{}time}
\PY{n}{model}\PY{p}{,} \PY{n}{cl\PYZus{}model} \PY{o}{=} \PY{n}{myKMeans}\PY{p}{(}\PY{l+m+mi}{15}\PY{p}{,} \PY{n}{data}\PY{p}{,} \PY{l+m+mi}{1}\PY{p}{,} \PY{l+s+s1}{\PYZsq{}}\PY{l+s+s1}{KMeans++}\PY{l+s+s1}{\PYZsq{}}\PY{p}{)}
\end{Verbatim}
\end{tcolorbox}

    \begin{Verbatim}[commandchars=\\\{\}]


NUMBER OF CLUSTERS: 15




Визуализация данных
\end{Verbatim}

    \begin{center}
    \adjustimage{max size={0.9\linewidth}{0.9\paperheight}}{output_11_1.png}
    \end{center}
    { \hspace*{\fill} \\}
    
    \begin{Verbatim}[commandchars=\\\{\}]


Визуализация центроидов
\end{Verbatim}

    \begin{center}
    \adjustimage{max size={0.9\linewidth}{0.9\paperheight}}{output_11_3.png}
    \end{center}
    { \hspace*{\fill} \\}
    
    \begin{Verbatim}[commandchars=\\\{\}]


Визуализация к какому классу относятся объкты после инициализации центроидов
\end{Verbatim}

    \begin{center}
    \adjustimage{max size={0.9\linewidth}{0.9\paperheight}}{output_11_5.png}
    \end{center}
    { \hspace*{\fill} \\}
    
    \begin{Verbatim}[commandchars=\\\{\}]


Визуализация после одного обновления положения центроидов
\end{Verbatim}

    \begin{center}
    \adjustimage{max size={0.9\linewidth}{0.9\paperheight}}{output_11_7.png}
    \end{center}
    { \hspace*{\fill} \\}
    
    \begin{Verbatim}[commandchars=\\\{\}]


Визуализация окончательного результата:
\end{Verbatim}

    \begin{center}
    \adjustimage{max size={0.9\linewidth}{0.9\paperheight}}{output_11_9.png}
    \end{center}
    { \hspace*{\fill} \\}
    
    \begin{Verbatim}[commandchars=\\\{\}]
CPU times: user 8.19 s, sys: 31.3 ms, total: 8.22 s
Wall time: 8.23 s
\end{Verbatim}

    \hypertarget{ux43dux430-ux43fux440ux438ux43cux435ux440ux435-ux432ux44bux448ux435-ux432ux438ux434ux43dux43e-ux44fux432ux43dux43eux435-ux43fux440ux435ux438ux43cux443ux449ux435ux441ux442ux432ux43e-ux43cux435ux442ux43eux434ux430-c-ux43fux43eux43cux43eux449ux44cux44e-ux43aux43eux442ux43eux440ux43eux433ux43e-ux446ux435ux43dux442ux440ux43eux438ux434ux44b-ux438ux43dux438ux446ux438ux430ux43bux438ux437ux438ux440ux443ux44eux442ux441ux44f-ux442ux430ux43aux438ux43c-ux43eux431ux440ux430ux437ux43eux43c-ux447ux442ux43eux431ux44b-ux43aux430ux436ux434ux44bux439-ux43dux43eux432ux44bux439-ux446ux435ux43dux442ux440ux43eux438ux434-ux43dux430ux445ux43eux434ux438ux43bux441ux44f-ux43dux430-ux43cux430ux43aux441ux438ux43cux430ux43bux44cux43dux43eux43c-ux440ux430ux441ux441ux442ux43eux44fux43dux438ux438-ux43eux442-ux43eux441ux442ux430ux43bux44cux43dux44bux445}{%
\subsubsection{На примере выше видно явное преимущество метода, c
помощью которого центроиды инициализируются таким образом, чтобы каждый
новый центроид находился на максимальном расстоянии от
остальных}\label{ux43dux430-ux43fux440ux438ux43cux435ux440ux435-ux432ux44bux448ux435-ux432ux438ux434ux43dux43e-ux44fux432ux43dux43eux435-ux43fux440ux435ux438ux43cux443ux449ux435ux441ux442ux432ux43e-ux43cux435ux442ux43eux434ux430-c-ux43fux43eux43cux43eux449ux44cux44e-ux43aux43eux442ux43eux440ux43eux433ux43e-ux446ux435ux43dux442ux440ux43eux438ux434ux44b-ux438ux43dux438ux446ux438ux430ux43bux438ux437ux438ux440ux443ux44eux442ux441ux44f-ux442ux430ux43aux438ux43c-ux43eux431ux440ux430ux437ux43eux43c-ux447ux442ux43eux431ux44b-ux43aux430ux436ux434ux44bux439-ux43dux43eux432ux44bux439-ux446ux435ux43dux442ux440ux43eux438ux434-ux43dux430ux445ux43eux434ux438ux43bux441ux44f-ux43dux430-ux43cux430ux43aux441ux438ux43cux430ux43bux44cux43dux43eux43c-ux440ux430ux441ux441ux442ux43eux44fux43dux438ux438-ux43eux442-ux43eux441ux442ux430ux43bux44cux43dux44bux445}}

    \hypertarget{section}{%
\subsection{------------}\label{section}}

\hypertarget{ux432ux438ux437ux443ux430ux43bux438ux437ux438ux440ux443ux435ux43c-ux43cux430ux442ux440ux438ux446ux443-ux43fux43eux43fux430ux440ux43dux44bux445-ux440ux430ux441ux441ux442ux43eux44fux43dux438ux439}{%
\subsubsection{Визуализируем матрицу попарных
расстояний}\label{ux432ux438ux437ux443ux430ux43bux438ux437ux438ux440ux443ux435ux43c-ux43cux430ux442ux440ux438ux446ux443-ux43fux43eux43fux430ux440ux43dux44bux445-ux440ux430ux441ux441ux442ux43eux44fux43dux438ux439}}

\hypertarget{dataset-ux440ux430ux441ux441ux43cux430ux442ux440ux438ux432ux430ux435ux43cux44bux439-ux43dux430ux43cux438-ux432-ux43dux430ux441ux442ux43eux44fux449ux438ux439-ux43cux43eux43cux435ux43dux442-ux441ux43aux430ux447ux435ux43d-ux438ux437-ux438ux43dux442ux435ux440ux43dux435ux442ux430-ux43fux43eux44dux442ux43eux43cux443-ux43cux430ux442ux440ux438ux446ux430-ux443ux43fux43eux440ux44fux434ux43eux447ux435ux43dux430-ux443ux436ux435-ux438ux437ux43dux430ux447ux430ux43bux44cux43dux43e}{%
\paragraph{Dataset рассматриваемый нами в настоящий момент скачен из
интернета, поэтому матрица упорядочена уже
изначально}\label{dataset-ux440ux430ux441ux441ux43cux430ux442ux440ux438ux432ux430ux435ux43cux44bux439-ux43dux430ux43cux438-ux432-ux43dux430ux441ux442ux43eux44fux449ux438ux439-ux43cux43eux43cux435ux43dux442-ux441ux43aux430ux447ux435ux43d-ux438ux437-ux438ux43dux442ux435ux440ux43dux435ux442ux430-ux43fux43eux44dux442ux43eux43cux443-ux43cux430ux442ux440ux438ux446ux430-ux443ux43fux43eux440ux44fux434ux43eux447ux435ux43dux430-ux443ux436ux435-ux438ux437ux43dux430ux447ux430ux43bux44cux43dux43e}}

    \begin{tcolorbox}[breakable, size=fbox, boxrule=1pt, pad at break*=1mm,colback=cellbackground, colframe=cellborder]
\prompt{In}{incolor}{192}{\hspace{4pt}}
\begin{Verbatim}[commandchars=\\\{\}]
\PY{n}{visualization}\PY{p}{(}\PY{n}{data}\PY{p}{,} \PY{n}{model}\PY{p}{,} \PY{n}{cl\PYZus{}model}\PY{p}{)}
\end{Verbatim}
\end{tcolorbox}

    \begin{center}
    \adjustimage{max size={0.9\linewidth}{0.9\paperheight}}{output_14_0.png}
    \end{center}
    { \hspace*{\fill} \\}
    
    \begin{center}
    \adjustimage{max size={0.9\linewidth}{0.9\paperheight}}{output_14_1.png}
    \end{center}
    { \hspace*{\fill} \\}
    
    \begin{center}
    \adjustimage{max size={0.9\linewidth}{0.9\paperheight}}{output_14_2.png}
    \end{center}
    { \hspace*{\fill} \\}
    
    \begin{center}
    \adjustimage{max size={0.9\linewidth}{0.9\paperheight}}{output_14_3.png}
    \end{center}
    { \hspace*{\fill} \\}
    
    \hypertarget{ux440ux430ux441ux441ux43cux43eux442ux440ux438ux43c-ux435ux449ux435-ux43dux435ux441ux43aux43eux43bux44cux43aux43e-sklearn-sampeux43eux432}{%
\subsubsection{Рассмотрим еще несколько sklearn
sampe'ов}\label{ux440ux430ux441ux441ux43cux43eux442ux440ux438ux43c-ux435ux449ux435-ux43dux435ux441ux43aux43eux43bux44cux43aux43e-sklearn-sampeux43eux432}}

\hypertarget{ux43fux43eux441ux43cux43eux442ux440ux438ux43c-ux43dux430-ux43cux430ux442ux440ux438ux446ux443-ux43fux43eux43fux430ux440ux43dux44bux445-ux440ux430ux441ux441ux442ux43eux44fux43dux438ux439-ux434ux43e-ux443ux43fux43eux440ux44fux434ux43eux447ux438ux432ux430ux43dux438ux44f-ux438-ux43fux43eux441ux43bux435}{%
\paragraph{Посмотрим на матрицу попарных расстояний до упорядочивания и
после}\label{ux43fux43eux441ux43cux43eux442ux440ux438ux43c-ux43dux430-ux43cux430ux442ux440ux438ux446ux443-ux43fux43eux43fux430ux440ux43dux44bux445-ux440ux430ux441ux441ux442ux43eux44fux43dux438ux439-ux434ux43e-ux443ux43fux43eux440ux44fux434ux43eux447ux438ux432ux430ux43dux438ux44f-ux438-ux43fux43eux441ux43bux435}}

\hypertarget{ux43dux430ux439ux434ux435ux43c-ux437ux430ux432ux438ux441ux438ux43cux43eux441ux442ux44c-ux432ux440ux435ux43cux435ux43dux438-ux43eux431ux443ux447ux435ux43dux438ux44f-ux43eux442-ux43eux431ux44aux435ux43cux430-ux438-ux441ux43bux43eux436ux43dux43eux441ux442ux438-ux437ux430ux434ux430ux447ux438}{%
\paragraph{Найдем зависимость времени обучения от объема и сложности
задачи}\label{ux43dux430ux439ux434ux435ux43c-ux437ux430ux432ux438ux441ux438ux43cux43eux441ux442ux44c-ux432ux440ux435ux43cux435ux43dux438-ux43eux431ux443ux447ux435ux43dux438ux44f-ux43eux442-ux43eux431ux44aux435ux43cux430-ux438-ux441ux43bux43eux436ux43dux43eux441ux442ux438-ux437ux430ux434ux430ux447ux438}}

    \begin{tcolorbox}[breakable, size=fbox, boxrule=1pt, pad at break*=1mm,colback=cellbackground, colframe=cellborder]
\prompt{In}{incolor}{193}{\hspace{4pt}}
\begin{Verbatim}[commandchars=\\\{\}]
\PY{n}{data}\PY{p}{,} \PY{n}{d} \PY{o}{=}  \PY{n}{make\PYZus{}blobs}\PY{p}{(}\PY{n}{n\PYZus{}samples}\PY{o}{=} \PY{l+m+mi}{1000}\PY{p}{,} \PY{n}{n\PYZus{}features}\PY{o}{=} \PY{l+m+mi}{2}\PY{p}{,} \PY{n}{centers} \PY{o}{=} \PY{l+m+mi}{3}\PY{p}{,} \PY{n}{random\PYZus{}state}\PY{o}{=}\PY{l+m+mi}{123}\PY{p}{)}
\end{Verbatim}
\end{tcolorbox}

    \begin{tcolorbox}[breakable, size=fbox, boxrule=1pt, pad at break*=1mm,colback=cellbackground, colframe=cellborder]
\prompt{In}{incolor}{194}{\hspace{4pt}}
\begin{Verbatim}[commandchars=\\\{\}]
\PY{n}{elbow\PYZus{}method}\PY{p}{(}\PY{n}{data}\PY{p}{,} \PY{l+m+mi}{2}\PY{p}{,} \PY{l+m+mi}{10}\PY{p}{)}
\end{Verbatim}
\end{tcolorbox}

    \begin{Verbatim}[commandchars=\\\{\}]
Running comput. for 2 cluster(s)
Running comput. for 3 cluster(s)
Running comput. for 4 cluster(s)
Running comput. for 5 cluster(s)
Running comput. for 6 cluster(s)
Running comput. for 7 cluster(s)
Running comput. for 8 cluster(s)
Running comput. for 9 cluster(s)
\end{Verbatim}

    \begin{center}
    \adjustimage{max size={0.9\linewidth}{0.9\paperheight}}{output_17_1.png}
    \end{center}
    { \hspace*{\fill} \\}
    
    \begin{tcolorbox}[breakable, size=fbox, boxrule=1pt, pad at break*=1mm,colback=cellbackground, colframe=cellborder]
\prompt{In}{incolor}{199}{\hspace{4pt}}
\begin{Verbatim}[commandchars=\\\{\}]
\PY{o}{\PYZpc{}\PYZpc{}time}
\PY{n}{model}\PY{p}{,} \PY{n}{cl\PYZus{}model} \PY{o}{=} \PY{n}{myKMeans}\PY{p}{(}\PY{l+m+mi}{5}\PY{p}{,} \PY{n}{data}\PY{p}{,} \PY{l+m+mi}{1}\PY{p}{,} \PY{l+s+s1}{\PYZsq{}}\PY{l+s+s1}{ }\PY{l+s+s1}{\PYZsq{}}\PY{p}{)}
\end{Verbatim}
\end{tcolorbox}

    \begin{Verbatim}[commandchars=\\\{\}]


NUMBER OF CLUSTERS: 5




Визуализация данных
\end{Verbatim}

    \begin{center}
    \adjustimage{max size={0.9\linewidth}{0.9\paperheight}}{output_18_1.png}
    \end{center}
    { \hspace*{\fill} \\}
    
    \begin{Verbatim}[commandchars=\\\{\}]


Визуализация центроидов
\end{Verbatim}

    \begin{center}
    \adjustimage{max size={0.9\linewidth}{0.9\paperheight}}{output_18_3.png}
    \end{center}
    { \hspace*{\fill} \\}
    
    \begin{Verbatim}[commandchars=\\\{\}]


Визуализация к какому классу относятся объкты после инициализации центроидов
\end{Verbatim}

    \begin{center}
    \adjustimage{max size={0.9\linewidth}{0.9\paperheight}}{output_18_5.png}
    \end{center}
    { \hspace*{\fill} \\}
    
    \begin{Verbatim}[commandchars=\\\{\}]


Визуализация после одного обновления положения центроидов
\end{Verbatim}

    \begin{center}
    \adjustimage{max size={0.9\linewidth}{0.9\paperheight}}{output_18_7.png}
    \end{center}
    { \hspace*{\fill} \\}
    
    \begin{Verbatim}[commandchars=\\\{\}]


Визуализация окончательного результата:
\end{Verbatim}

    \begin{center}
    \adjustimage{max size={0.9\linewidth}{0.9\paperheight}}{output_18_9.png}
    \end{center}
    { \hspace*{\fill} \\}
    
    \begin{Verbatim}[commandchars=\\\{\}]
CPU times: user 2.9 s, sys: 64 ms, total: 2.96 s
Wall time: 3.07 s
\end{Verbatim}

    \begin{tcolorbox}[breakable, size=fbox, boxrule=1pt, pad at break*=1mm,colback=cellbackground, colframe=cellborder]
\prompt{In}{incolor}{200}{\hspace{4pt}}
\begin{Verbatim}[commandchars=\\\{\}]
\PY{n}{visualization}\PY{p}{(}\PY{n}{data}\PY{p}{,} \PY{n}{model}\PY{p}{,} \PY{n}{cl\PYZus{}model}\PY{p}{)}
\end{Verbatim}
\end{tcolorbox}

    \begin{center}
    \adjustimage{max size={0.9\linewidth}{0.9\paperheight}}{output_19_0.png}
    \end{center}
    { \hspace*{\fill} \\}
    
    \begin{center}
    \adjustimage{max size={0.9\linewidth}{0.9\paperheight}}{output_19_1.png}
    \end{center}
    { \hspace*{\fill} \\}
    
    \begin{center}
    \adjustimage{max size={0.9\linewidth}{0.9\paperheight}}{output_19_2.png}
    \end{center}
    { \hspace*{\fill} \\}
    
    \begin{center}
    \adjustimage{max size={0.9\linewidth}{0.9\paperheight}}{output_19_3.png}
    \end{center}
    { \hspace*{\fill} \\}
    
    \begin{tcolorbox}[breakable, size=fbox, boxrule=1pt, pad at break*=1mm,colback=cellbackground, colframe=cellborder]
\prompt{In}{incolor}{201}{\hspace{4pt}}
\begin{Verbatim}[commandchars=\\\{\}]
\PY{o}{\PYZpc{}\PYZpc{}time}
\PY{n}{model}\PY{p}{,} \PY{n}{cl\PYZus{}model} \PY{o}{=} \PY{n}{myKMeans}\PY{p}{(}\PY{l+m+mi}{5}\PY{p}{,} \PY{n}{data}\PY{p}{,} \PY{l+m+mi}{1}\PY{p}{,} \PY{l+s+s1}{\PYZsq{}}\PY{l+s+s1}{KMeans++}\PY{l+s+s1}{\PYZsq{}}\PY{p}{)}
\end{Verbatim}
\end{tcolorbox}

    \begin{Verbatim}[commandchars=\\\{\}]


NUMBER OF CLUSTERS: 5




Визуализация данных
\end{Verbatim}

    \begin{center}
    \adjustimage{max size={0.9\linewidth}{0.9\paperheight}}{output_20_1.png}
    \end{center}
    { \hspace*{\fill} \\}
    
    \begin{Verbatim}[commandchars=\\\{\}]


Визуализация центроидов
\end{Verbatim}

    \begin{center}
    \adjustimage{max size={0.9\linewidth}{0.9\paperheight}}{output_20_3.png}
    \end{center}
    { \hspace*{\fill} \\}
    
    \begin{Verbatim}[commandchars=\\\{\}]


Визуализация к какому классу относятся объкты после инициализации центроидов
\end{Verbatim}

    \begin{center}
    \adjustimage{max size={0.9\linewidth}{0.9\paperheight}}{output_20_5.png}
    \end{center}
    { \hspace*{\fill} \\}
    
    \begin{Verbatim}[commandchars=\\\{\}]


Визуализация после одного обновления положения центроидов
\end{Verbatim}

    \begin{center}
    \adjustimage{max size={0.9\linewidth}{0.9\paperheight}}{output_20_7.png}
    \end{center}
    { \hspace*{\fill} \\}
    
    \begin{Verbatim}[commandchars=\\\{\}]


Визуализация окончательного результата:
\end{Verbatim}

    \begin{center}
    \adjustimage{max size={0.9\linewidth}{0.9\paperheight}}{output_20_9.png}
    \end{center}
    { \hspace*{\fill} \\}
    
    \begin{Verbatim}[commandchars=\\\{\}]
CPU times: user 1.98 s, sys: 54.3 ms, total: 2.03 s
Wall time: 2.11 s
\end{Verbatim}

    \begin{tcolorbox}[breakable, size=fbox, boxrule=1pt, pad at break*=1mm,colback=cellbackground, colframe=cellborder]
\prompt{In}{incolor}{202}{\hspace{4pt}}
\begin{Verbatim}[commandchars=\\\{\}]
\PY{n}{visualization}\PY{p}{(}\PY{n}{data}\PY{p}{,} \PY{n}{model}\PY{p}{,} \PY{n}{cl\PYZus{}model}\PY{p}{)}
\end{Verbatim}
\end{tcolorbox}

    \begin{center}
    \adjustimage{max size={0.9\linewidth}{0.9\paperheight}}{output_21_0.png}
    \end{center}
    { \hspace*{\fill} \\}
    
    \begin{center}
    \adjustimage{max size={0.9\linewidth}{0.9\paperheight}}{output_21_1.png}
    \end{center}
    { \hspace*{\fill} \\}
    
    \begin{center}
    \adjustimage{max size={0.9\linewidth}{0.9\paperheight}}{output_21_2.png}
    \end{center}
    { \hspace*{\fill} \\}
    
    \begin{center}
    \adjustimage{max size={0.9\linewidth}{0.9\paperheight}}{output_21_3.png}
    \end{center}
    { \hspace*{\fill} \\}
    
    \begin{tcolorbox}[breakable, size=fbox, boxrule=1pt, pad at break*=1mm,colback=cellbackground, colframe=cellborder]
\prompt{In}{incolor}{206}{\hspace{4pt}}
\begin{Verbatim}[commandchars=\\\{\}]
\PY{n}{data}\PY{p}{,} \PY{n}{d} \PY{o}{=}  \PY{n}{make\PYZus{}blobs}\PY{p}{(}\PY{n}{n\PYZus{}samples}\PY{o}{=} \PY{l+m+mi}{1000}\PY{p}{,} \PY{n}{n\PYZus{}features}\PY{o}{=} \PY{l+m+mi}{2}\PY{p}{,} \PY{n}{centers} \PY{o}{=} \PY{l+m+mi}{5}\PY{p}{,} \PY{n}{random\PYZus{}state}\PY{o}{=}\PY{l+m+mi}{123}\PY{p}{)}
\end{Verbatim}
\end{tcolorbox}

    \begin{tcolorbox}[breakable, size=fbox, boxrule=1pt, pad at break*=1mm,colback=cellbackground, colframe=cellborder]
\prompt{In}{incolor}{211}{\hspace{4pt}}
\begin{Verbatim}[commandchars=\\\{\}]
\PY{n}{elbow\PYZus{}method}\PY{p}{(}\PY{n}{data}\PY{p}{,} \PY{l+m+mi}{2}\PY{p}{,} \PY{l+m+mi}{10}\PY{p}{)}
\end{Verbatim}
\end{tcolorbox}

    \begin{Verbatim}[commandchars=\\\{\}]
Running comput. for 2 cluster(s)
Running comput. for 3 cluster(s)
Running comput. for 4 cluster(s)
Running comput. for 5 cluster(s)
Running comput. for 6 cluster(s)
Running comput. for 7 cluster(s)
Running comput. for 8 cluster(s)
Running comput. for 9 cluster(s)
\end{Verbatim}

    \begin{center}
    \adjustimage{max size={0.9\linewidth}{0.9\paperheight}}{output_23_1.png}
    \end{center}
    { \hspace*{\fill} \\}
    
    \begin{tcolorbox}[breakable, size=fbox, boxrule=1pt, pad at break*=1mm,colback=cellbackground, colframe=cellborder]
\prompt{In}{incolor}{212}{\hspace{4pt}}
\begin{Verbatim}[commandchars=\\\{\}]
\PY{o}{\PYZpc{}\PYZpc{}time}
\PY{n}{model}\PY{p}{,} \PY{n}{cl\PYZus{}model} \PY{o}{=} \PY{n}{myKMeans}\PY{p}{(}\PY{l+m+mi}{3}\PY{p}{,} \PY{n}{data}\PY{p}{,} \PY{l+m+mi}{1}\PY{p}{,} \PY{l+s+s1}{\PYZsq{}}\PY{l+s+s1}{ }\PY{l+s+s1}{\PYZsq{}}\PY{p}{)}
\end{Verbatim}
\end{tcolorbox}

    \begin{Verbatim}[commandchars=\\\{\}]


NUMBER OF CLUSTERS: 3




Визуализация данных
\end{Verbatim}

    \begin{center}
    \adjustimage{max size={0.9\linewidth}{0.9\paperheight}}{output_24_1.png}
    \end{center}
    { \hspace*{\fill} \\}
    
    \begin{Verbatim}[commandchars=\\\{\}]


Визуализация центроидов
\end{Verbatim}

    \begin{center}
    \adjustimage{max size={0.9\linewidth}{0.9\paperheight}}{output_24_3.png}
    \end{center}
    { \hspace*{\fill} \\}
    
    \begin{Verbatim}[commandchars=\\\{\}]


Визуализация к какому классу относятся объкты после инициализации центроидов
\end{Verbatim}

    \begin{center}
    \adjustimage{max size={0.9\linewidth}{0.9\paperheight}}{output_24_5.png}
    \end{center}
    { \hspace*{\fill} \\}
    
    \begin{Verbatim}[commandchars=\\\{\}]


Визуализация после одного обновления положения центроидов
\end{Verbatim}

    \begin{center}
    \adjustimage{max size={0.9\linewidth}{0.9\paperheight}}{output_24_7.png}
    \end{center}
    { \hspace*{\fill} \\}
    
    \begin{Verbatim}[commandchars=\\\{\}]


Визуализация окончательного результата:
\end{Verbatim}

    \begin{center}
    \adjustimage{max size={0.9\linewidth}{0.9\paperheight}}{output_24_9.png}
    \end{center}
    { \hspace*{\fill} \\}
    
    \begin{Verbatim}[commandchars=\\\{\}]
CPU times: user 1.37 s, sys: 47.1 ms, total: 1.42 s
Wall time: 1.49 s
\end{Verbatim}

    \begin{tcolorbox}[breakable, size=fbox, boxrule=1pt, pad at break*=1mm,colback=cellbackground, colframe=cellborder]
\prompt{In}{incolor}{209}{\hspace{4pt}}
\begin{Verbatim}[commandchars=\\\{\}]
\PY{n}{visualization}\PY{p}{(}\PY{n}{data}\PY{p}{,} \PY{n}{model}\PY{p}{,} \PY{n}{cl\PYZus{}model}\PY{p}{)}
\end{Verbatim}
\end{tcolorbox}

    \begin{center}
    \adjustimage{max size={0.9\linewidth}{0.9\paperheight}}{output_25_0.png}
    \end{center}
    { \hspace*{\fill} \\}
    
    \begin{center}
    \adjustimage{max size={0.9\linewidth}{0.9\paperheight}}{output_25_1.png}
    \end{center}
    { \hspace*{\fill} \\}
    
    \begin{center}
    \adjustimage{max size={0.9\linewidth}{0.9\paperheight}}{output_25_2.png}
    \end{center}
    { \hspace*{\fill} \\}
    
    \begin{center}
    \adjustimage{max size={0.9\linewidth}{0.9\paperheight}}{output_25_3.png}
    \end{center}
    { \hspace*{\fill} \\}
    
    \begin{tcolorbox}[breakable, size=fbox, boxrule=1pt, pad at break*=1mm,colback=cellbackground, colframe=cellborder]
\prompt{In}{incolor}{210}{\hspace{4pt}}
\begin{Verbatim}[commandchars=\\\{\}]
\PY{o}{\PYZpc{}\PYZpc{}time}
\PY{n}{model}\PY{p}{,} \PY{n}{cl\PYZus{}model} \PY{o}{=} \PY{n}{myKMeans}\PY{p}{(}\PY{l+m+mi}{5}\PY{p}{,} \PY{n}{data}\PY{p}{,} \PY{l+m+mi}{1}\PY{p}{,} \PY{l+s+s1}{\PYZsq{}}\PY{l+s+s1}{KMeans++}\PY{l+s+s1}{\PYZsq{}}\PY{p}{)}
\end{Verbatim}
\end{tcolorbox}

    \begin{Verbatim}[commandchars=\\\{\}]


NUMBER OF CLUSTERS: 5




Визуализация данных
\end{Verbatim}

    \begin{center}
    \adjustimage{max size={0.9\linewidth}{0.9\paperheight}}{output_26_1.png}
    \end{center}
    { \hspace*{\fill} \\}
    
    \begin{Verbatim}[commandchars=\\\{\}]


Визуализация центроидов
\end{Verbatim}

    \begin{center}
    \adjustimage{max size={0.9\linewidth}{0.9\paperheight}}{output_26_3.png}
    \end{center}
    { \hspace*{\fill} \\}
    
    \begin{Verbatim}[commandchars=\\\{\}]


Визуализация к какому классу относятся объкты после инициализации центроидов
\end{Verbatim}

    \begin{center}
    \adjustimage{max size={0.9\linewidth}{0.9\paperheight}}{output_26_5.png}
    \end{center}
    { \hspace*{\fill} \\}
    
    \begin{Verbatim}[commandchars=\\\{\}]


Визуализация после одного обновления положения центроидов
\end{Verbatim}

    \begin{center}
    \adjustimage{max size={0.9\linewidth}{0.9\paperheight}}{output_26_7.png}
    \end{center}
    { \hspace*{\fill} \\}
    
    \begin{Verbatim}[commandchars=\\\{\}]


Визуализация окончательного результата:
\end{Verbatim}

    \begin{center}
    \adjustimage{max size={0.9\linewidth}{0.9\paperheight}}{output_26_9.png}
    \end{center}
    { \hspace*{\fill} \\}
    
    \begin{Verbatim}[commandchars=\\\{\}]
CPU times: user 1.66 s, sys: 55 ms, total: 1.72 s
Wall time: 1.83 s
\end{Verbatim}

    \begin{tcolorbox}[breakable, size=fbox, boxrule=1pt, pad at break*=1mm,colback=cellbackground, colframe=cellborder]
\prompt{In}{incolor}{213}{\hspace{4pt}}
\begin{Verbatim}[commandchars=\\\{\}]
\PY{n}{visualization}\PY{p}{(}\PY{n}{data}\PY{p}{,} \PY{n}{model}\PY{p}{,} \PY{n}{cl\PYZus{}model}\PY{p}{)}
\end{Verbatim}
\end{tcolorbox}

    \begin{center}
    \adjustimage{max size={0.9\linewidth}{0.9\paperheight}}{output_27_0.png}
    \end{center}
    { \hspace*{\fill} \\}
    
    \begin{center}
    \adjustimage{max size={0.9\linewidth}{0.9\paperheight}}{output_27_1.png}
    \end{center}
    { \hspace*{\fill} \\}
    
    \begin{center}
    \adjustimage{max size={0.9\linewidth}{0.9\paperheight}}{output_27_2.png}
    \end{center}
    { \hspace*{\fill} \\}
    
    \begin{center}
    \adjustimage{max size={0.9\linewidth}{0.9\paperheight}}{output_27_3.png}
    \end{center}
    { \hspace*{\fill} \\}
    
    \hypertarget{ux43dux430-ux434ux430ux43dux43dux43eux43c-ux43fux440ux438ux43cux435ux440ux435-ux43fux440ux438-random.seed123-ux43fux43eux43bux443ux447ux430ux435ux442ux441ux44f-ux447ux442ux43e-kmeans-ux440ux430ux431ux43eux442ux430ux435ux442-ux43bux443ux447ux448ux435}{%
\subsubsection{На данном примере при random.seed(123) получается, что
kMeans работает
лучше}\label{ux43dux430-ux434ux430ux43dux43dux43eux43c-ux43fux440ux438ux43cux435ux440ux435-ux43fux440ux438-random.seed123-ux43fux43eux43bux443ux447ux430ux435ux442ux441ux44f-ux447ux442ux43e-kmeans-ux440ux430ux431ux43eux442ux430ux435ux442-ux43bux443ux447ux448ux435}}

    \begin{tcolorbox}[breakable, size=fbox, boxrule=1pt, pad at break*=1mm,colback=cellbackground, colframe=cellborder]
\prompt{In}{incolor}{224}{\hspace{4pt}}
\begin{Verbatim}[commandchars=\\\{\}]
\PY{n}{data}\PY{p}{,} \PY{n}{d} \PY{o}{=}  \PY{n}{make\PYZus{}blobs}\PY{p}{(}\PY{n}{n\PYZus{}samples}\PY{o}{=} \PY{l+m+mi}{5000}\PY{p}{,} \PY{n}{n\PYZus{}features}\PY{o}{=} \PY{l+m+mi}{2}\PY{p}{,} \PY{n}{centers} \PY{o}{=} \PY{l+m+mi}{5}\PY{p}{,} \PY{n}{random\PYZus{}state}\PY{o}{=}\PY{l+m+mi}{123}\PY{p}{)}
\end{Verbatim}
\end{tcolorbox}

    \begin{tcolorbox}[breakable, size=fbox, boxrule=1pt, pad at break*=1mm,colback=cellbackground, colframe=cellborder]
\prompt{In}{incolor}{227}{\hspace{4pt}}
\begin{Verbatim}[commandchars=\\\{\}]
\PY{o}{\PYZpc{}\PYZpc{}time}
\PY{n}{elbow\PYZus{}method}\PY{p}{(}\PY{n}{data}\PY{p}{,} \PY{l+m+mi}{2}\PY{p}{,} \PY{l+m+mi}{10}\PY{p}{)}
\end{Verbatim}
\end{tcolorbox}

    \begin{Verbatim}[commandchars=\\\{\}]
Running comput. for 2 cluster(s)
Running comput. for 3 cluster(s)
Running comput. for 4 cluster(s)
Running comput. for 5 cluster(s)
Running comput. for 6 cluster(s)
Running comput. for 7 cluster(s)
Running comput. for 8 cluster(s)
Running comput. for 9 cluster(s)
CPU times: user 44.1 s, sys: 903 ms, total: 45 s
Wall time: 42.8 s
\end{Verbatim}

    \begin{center}
    \adjustimage{max size={0.9\linewidth}{0.9\paperheight}}{output_30_1.png}
    \end{center}
    { \hspace*{\fill} \\}
    
    \begin{tcolorbox}[breakable, size=fbox, boxrule=1pt, pad at break*=1mm,colback=cellbackground, colframe=cellborder]
\prompt{In}{incolor}{216}{\hspace{4pt}}
\begin{Verbatim}[commandchars=\\\{\}]
\PY{o}{\PYZpc{}\PYZpc{}time}
\PY{n}{model}\PY{p}{,} \PY{n}{cl\PYZus{}model} \PY{o}{=} \PY{n}{myKMeans}\PY{p}{(}\PY{l+m+mi}{5}\PY{p}{,} \PY{n}{data}\PY{p}{,} \PY{l+m+mi}{1}\PY{p}{,} \PY{l+s+s1}{\PYZsq{}}\PY{l+s+s1}{ }\PY{l+s+s1}{\PYZsq{}}\PY{p}{)}
\end{Verbatim}
\end{tcolorbox}

    \begin{Verbatim}[commandchars=\\\{\}]


NUMBER OF CLUSTERS: 5




Визуализация данных
\end{Verbatim}

    \begin{center}
    \adjustimage{max size={0.9\linewidth}{0.9\paperheight}}{output_31_1.png}
    \end{center}
    { \hspace*{\fill} \\}
    
    \begin{Verbatim}[commandchars=\\\{\}]


Визуализация центроидов
\end{Verbatim}

    \begin{center}
    \adjustimage{max size={0.9\linewidth}{0.9\paperheight}}{output_31_3.png}
    \end{center}
    { \hspace*{\fill} \\}
    
    \begin{Verbatim}[commandchars=\\\{\}]


Визуализация к какому классу относятся объкты после инициализации центроидов
\end{Verbatim}

    \begin{center}
    \adjustimage{max size={0.9\linewidth}{0.9\paperheight}}{output_31_5.png}
    \end{center}
    { \hspace*{\fill} \\}
    
    \begin{Verbatim}[commandchars=\\\{\}]


Визуализация после одного обновления положения центроидов
\end{Verbatim}

    \begin{center}
    \adjustimage{max size={0.9\linewidth}{0.9\paperheight}}{output_31_7.png}
    \end{center}
    { \hspace*{\fill} \\}
    
    \begin{Verbatim}[commandchars=\\\{\}]


Визуализация окончательного результата:
\end{Verbatim}

    \begin{center}
    \adjustimage{max size={0.9\linewidth}{0.9\paperheight}}{output_31_9.png}
    \end{center}
    { \hspace*{\fill} \\}
    
    \begin{Verbatim}[commandchars=\\\{\}]
CPU times: user 4.71 s, sys: 111 ms, total: 4.82 s
Wall time: 5.27 s
\end{Verbatim}

    \begin{tcolorbox}[breakable, size=fbox, boxrule=1pt, pad at break*=1mm,colback=cellbackground, colframe=cellborder]
\prompt{In}{incolor}{217}{\hspace{4pt}}
\begin{Verbatim}[commandchars=\\\{\}]
\PY{o}{\PYZpc{}\PYZpc{}time}
\PY{n}{model}\PY{p}{,} \PY{n}{cl\PYZus{}model} \PY{o}{=} \PY{n}{myKMeans}\PY{p}{(}\PY{l+m+mi}{5}\PY{p}{,} \PY{n}{data}\PY{p}{,} \PY{l+m+mi}{1}\PY{p}{,} \PY{l+s+s1}{\PYZsq{}}\PY{l+s+s1}{KMeans++}\PY{l+s+s1}{\PYZsq{}}\PY{p}{)}
\end{Verbatim}
\end{tcolorbox}

    \begin{Verbatim}[commandchars=\\\{\}]


NUMBER OF CLUSTERS: 5




Визуализация данных
\end{Verbatim}

    \begin{center}
    \adjustimage{max size={0.9\linewidth}{0.9\paperheight}}{output_32_1.png}
    \end{center}
    { \hspace*{\fill} \\}
    
    \begin{Verbatim}[commandchars=\\\{\}]


Визуализация центроидов
\end{Verbatim}

    \begin{center}
    \adjustimage{max size={0.9\linewidth}{0.9\paperheight}}{output_32_3.png}
    \end{center}
    { \hspace*{\fill} \\}
    
    \begin{Verbatim}[commandchars=\\\{\}]


Визуализация к какому классу относятся объкты после инициализации центроидов
\end{Verbatim}

    \begin{center}
    \adjustimage{max size={0.9\linewidth}{0.9\paperheight}}{output_32_5.png}
    \end{center}
    { \hspace*{\fill} \\}
    
    \begin{Verbatim}[commandchars=\\\{\}]


Визуализация после одного обновления положения центроидов
\end{Verbatim}

    \begin{center}
    \adjustimage{max size={0.9\linewidth}{0.9\paperheight}}{output_32_7.png}
    \end{center}
    { \hspace*{\fill} \\}
    
    \begin{Verbatim}[commandchars=\\\{\}]


Визуализация окончательного результата:
\end{Verbatim}

    \begin{center}
    \adjustimage{max size={0.9\linewidth}{0.9\paperheight}}{output_32_9.png}
    \end{center}
    { \hspace*{\fill} \\}
    
    \begin{Verbatim}[commandchars=\\\{\}]
CPU times: user 4.26 s, sys: 97.2 ms, total: 4.36 s
Wall time: 4.68 s
\end{Verbatim}

    \begin{tcolorbox}[breakable, size=fbox, boxrule=1pt, pad at break*=1mm,colback=cellbackground, colframe=cellborder]
\prompt{In}{incolor}{228}{\hspace{4pt}}
\begin{Verbatim}[commandchars=\\\{\}]
\PY{n}{data}\PY{p}{,} \PY{n}{d} \PY{o}{=}  \PY{n}{make\PYZus{}blobs}\PY{p}{(}\PY{n}{n\PYZus{}samples}\PY{o}{=} \PY{l+m+mi}{30000}\PY{p}{,} \PY{n}{n\PYZus{}features}\PY{o}{=} \PY{l+m+mi}{2}\PY{p}{,} \PY{n}{centers} \PY{o}{=} \PY{l+m+mi}{5}\PY{p}{,} \PY{n}{random\PYZus{}state}\PY{o}{=}\PY{l+m+mi}{123}\PY{p}{)}
\end{Verbatim}
\end{tcolorbox}

    \begin{tcolorbox}[breakable, size=fbox, boxrule=1pt, pad at break*=1mm,colback=cellbackground, colframe=cellborder]
\prompt{In}{incolor}{229}{\hspace{4pt}}
\begin{Verbatim}[commandchars=\\\{\}]
\PY{o}{\PYZpc{}\PYZpc{}time}
\PY{n}{model}\PY{p}{,} \PY{n}{cl\PYZus{}model} \PY{o}{=} \PY{n}{myKMeans}\PY{p}{(}\PY{l+m+mi}{5}\PY{p}{,} \PY{n}{data}\PY{p}{,} \PY{l+m+mi}{1}\PY{p}{,} \PY{l+s+s1}{\PYZsq{}}\PY{l+s+s1}{ }\PY{l+s+s1}{\PYZsq{}}\PY{p}{)}
\end{Verbatim}
\end{tcolorbox}

    \begin{Verbatim}[commandchars=\\\{\}]


NUMBER OF CLUSTERS: 5




Визуализация данных
\end{Verbatim}

    \begin{center}
    \adjustimage{max size={0.9\linewidth}{0.9\paperheight}}{output_34_1.png}
    \end{center}
    { \hspace*{\fill} \\}
    
    \begin{Verbatim}[commandchars=\\\{\}]


Визуализация центроидов
\end{Verbatim}

    \begin{center}
    \adjustimage{max size={0.9\linewidth}{0.9\paperheight}}{output_34_3.png}
    \end{center}
    { \hspace*{\fill} \\}
    
    \begin{Verbatim}[commandchars=\\\{\}]


Визуализация к какому классу относятся объкты после инициализации центроидов
\end{Verbatim}

    \begin{center}
    \adjustimage{max size={0.9\linewidth}{0.9\paperheight}}{output_34_5.png}
    \end{center}
    { \hspace*{\fill} \\}
    
    \begin{Verbatim}[commandchars=\\\{\}]


Визуализация после одного обновления положения центроидов
\end{Verbatim}

    \begin{center}
    \adjustimage{max size={0.9\linewidth}{0.9\paperheight}}{output_34_7.png}
    \end{center}
    { \hspace*{\fill} \\}
    
    \begin{Verbatim}[commandchars=\\\{\}]


Визуализация окончательного результата:
\end{Verbatim}

    \begin{center}
    \adjustimage{max size={0.9\linewidth}{0.9\paperheight}}{output_34_9.png}
    \end{center}
    { \hspace*{\fill} \\}
    
    \begin{Verbatim}[commandchars=\\\{\}]
CPU times: user 19 s, sys: 113 ms, total: 19.2 s
Wall time: 19.3 s
\end{Verbatim}

    \begin{tcolorbox}[breakable, size=fbox, boxrule=1pt, pad at break*=1mm,colback=cellbackground, colframe=cellborder]
\prompt{In}{incolor}{230}{\hspace{4pt}}
\begin{Verbatim}[commandchars=\\\{\}]
\PY{o}{\PYZpc{}\PYZpc{}time}
\PY{n}{model}\PY{p}{,} \PY{n}{cl\PYZus{}model} \PY{o}{=} \PY{n}{myKMeans}\PY{p}{(}\PY{l+m+mi}{5}\PY{p}{,} \PY{n}{data}\PY{p}{,} \PY{l+m+mi}{1}\PY{p}{,} \PY{l+s+s1}{\PYZsq{}}\PY{l+s+s1}{KMeans++}\PY{l+s+s1}{\PYZsq{}}\PY{p}{)}
\end{Verbatim}
\end{tcolorbox}

    \begin{Verbatim}[commandchars=\\\{\}]


NUMBER OF CLUSTERS: 5




Визуализация данных
\end{Verbatim}

    \begin{center}
    \adjustimage{max size={0.9\linewidth}{0.9\paperheight}}{output_35_1.png}
    \end{center}
    { \hspace*{\fill} \\}
    
    \begin{Verbatim}[commandchars=\\\{\}]


Визуализация центроидов
\end{Verbatim}

    \begin{center}
    \adjustimage{max size={0.9\linewidth}{0.9\paperheight}}{output_35_3.png}
    \end{center}
    { \hspace*{\fill} \\}
    
    \begin{Verbatim}[commandchars=\\\{\}]


Визуализация к какому классу относятся объкты после инициализации центроидов
\end{Verbatim}

    \begin{center}
    \adjustimage{max size={0.9\linewidth}{0.9\paperheight}}{output_35_5.png}
    \end{center}
    { \hspace*{\fill} \\}
    
    \begin{Verbatim}[commandchars=\\\{\}]


Визуализация после одного обновления положения центроидов
\end{Verbatim}

    \begin{center}
    \adjustimage{max size={0.9\linewidth}{0.9\paperheight}}{output_35_7.png}
    \end{center}
    { \hspace*{\fill} \\}
    
    \begin{Verbatim}[commandchars=\\\{\}]


Визуализация окончательного результата:
\end{Verbatim}

    \begin{center}
    \adjustimage{max size={0.9\linewidth}{0.9\paperheight}}{output_35_9.png}
    \end{center}
    { \hspace*{\fill} \\}
    
    \begin{Verbatim}[commandchars=\\\{\}]
CPU times: user 16.9 s, sys: 90 ms, total: 17 s
Wall time: 17.3 s
\end{Verbatim}

    \begin{tcolorbox}[breakable, size=fbox, boxrule=1pt, pad at break*=1mm,colback=cellbackground, colframe=cellborder]
\prompt{In}{incolor}{238}{\hspace{4pt}}
\begin{Verbatim}[commandchars=\\\{\}]
\PY{n}{data}\PY{p}{,} \PY{n}{d} \PY{o}{=}  \PY{n}{make\PYZus{}blobs}\PY{p}{(}\PY{n}{n\PYZus{}samples}\PY{o}{=} \PY{l+m+mi}{5000}\PY{p}{,} \PY{n}{n\PYZus{}features}\PY{o}{=} \PY{l+m+mi}{2}\PY{p}{,} \PY{n}{centers} \PY{o}{=} \PY{l+m+mi}{15}\PY{p}{,} \PY{n}{random\PYZus{}state}\PY{o}{=}\PY{l+m+mi}{123}\PY{p}{)}
\end{Verbatim}
\end{tcolorbox}

    \begin{tcolorbox}[breakable, size=fbox, boxrule=1pt, pad at break*=1mm,colback=cellbackground, colframe=cellborder]
\prompt{In}{incolor}{239}{\hspace{4pt}}
\begin{Verbatim}[commandchars=\\\{\}]
\PY{o}{\PYZpc{}\PYZpc{}time}
\PY{n}{model}\PY{p}{,} \PY{n}{cl\PYZus{}model} \PY{o}{=} \PY{n}{myKMeans}\PY{p}{(}\PY{l+m+mi}{15}\PY{p}{,} \PY{n}{data}\PY{p}{,} \PY{l+m+mi}{1}\PY{p}{,} \PY{l+s+s1}{\PYZsq{}}\PY{l+s+s1}{ }\PY{l+s+s1}{\PYZsq{}}\PY{p}{)}
\end{Verbatim}
\end{tcolorbox}

    \begin{Verbatim}[commandchars=\\\{\}]


NUMBER OF CLUSTERS: 15




Визуализация данных
\end{Verbatim}

    \begin{center}
    \adjustimage{max size={0.9\linewidth}{0.9\paperheight}}{output_37_1.png}
    \end{center}
    { \hspace*{\fill} \\}
    
    \begin{Verbatim}[commandchars=\\\{\}]


Визуализация центроидов
\end{Verbatim}

    \begin{center}
    \adjustimage{max size={0.9\linewidth}{0.9\paperheight}}{output_37_3.png}
    \end{center}
    { \hspace*{\fill} \\}
    
    \begin{Verbatim}[commandchars=\\\{\}]


Визуализация к какому классу относятся объкты после инициализации центроидов
\end{Verbatim}

    \begin{center}
    \adjustimage{max size={0.9\linewidth}{0.9\paperheight}}{output_37_5.png}
    \end{center}
    { \hspace*{\fill} \\}
    
    \begin{Verbatim}[commandchars=\\\{\}]


Визуализация после одного обновления положения центроидов
\end{Verbatim}

    \begin{center}
    \adjustimage{max size={0.9\linewidth}{0.9\paperheight}}{output_37_7.png}
    \end{center}
    { \hspace*{\fill} \\}
    
    \begin{Verbatim}[commandchars=\\\{\}]


Визуализация окончательного результата:
\end{Verbatim}

    \begin{center}
    \adjustimage{max size={0.9\linewidth}{0.9\paperheight}}{output_37_9.png}
    \end{center}
    { \hspace*{\fill} \\}
    
    \begin{Verbatim}[commandchars=\\\{\}]
CPU times: user 27.6 s, sys: 388 ms, total: 28 s
Wall time: 30.5 s
\end{Verbatim}

    \begin{tcolorbox}[breakable, size=fbox, boxrule=1pt, pad at break*=1mm,colback=cellbackground, colframe=cellborder]
\prompt{In}{incolor}{240}{\hspace{4pt}}
\begin{Verbatim}[commandchars=\\\{\}]
\PY{o}{\PYZpc{}\PYZpc{}time}
\PY{n}{model}\PY{p}{,} \PY{n}{cl\PYZus{}model} \PY{o}{=} \PY{n}{myKMeans}\PY{p}{(}\PY{l+m+mi}{15}\PY{p}{,} \PY{n}{data}\PY{p}{,} \PY{l+m+mi}{1}\PY{p}{,} \PY{l+s+s1}{\PYZsq{}}\PY{l+s+s1}{KMeans++}\PY{l+s+s1}{\PYZsq{}}\PY{p}{)}
\end{Verbatim}
\end{tcolorbox}

    \begin{Verbatim}[commandchars=\\\{\}]


NUMBER OF CLUSTERS: 15




Визуализация данных
\end{Verbatim}

    \begin{center}
    \adjustimage{max size={0.9\linewidth}{0.9\paperheight}}{output_38_1.png}
    \end{center}
    { \hspace*{\fill} \\}
    
    \begin{Verbatim}[commandchars=\\\{\}]


Визуализация центроидов
\end{Verbatim}

    \begin{center}
    \adjustimage{max size={0.9\linewidth}{0.9\paperheight}}{output_38_3.png}
    \end{center}
    { \hspace*{\fill} \\}
    
    \begin{Verbatim}[commandchars=\\\{\}]


Визуализация к какому классу относятся объкты после инициализации центроидов
\end{Verbatim}

    \begin{center}
    \adjustimage{max size={0.9\linewidth}{0.9\paperheight}}{output_38_5.png}
    \end{center}
    { \hspace*{\fill} \\}
    
    \begin{Verbatim}[commandchars=\\\{\}]


Визуализация после одного обновления положения центроидов
\end{Verbatim}

    \begin{center}
    \adjustimage{max size={0.9\linewidth}{0.9\paperheight}}{output_38_7.png}
    \end{center}
    { \hspace*{\fill} \\}
    
    \begin{Verbatim}[commandchars=\\\{\}]


Визуализация окончательного результата:
\end{Verbatim}

    \begin{center}
    \adjustimage{max size={0.9\linewidth}{0.9\paperheight}}{output_38_9.png}
    \end{center}
    { \hspace*{\fill} \\}
    
    \begin{Verbatim}[commandchars=\\\{\}]
CPU times: user 13.6 s, sys: 168 ms, total: 13.8 s
Wall time: 15 s
\end{Verbatim}

    \begin{tcolorbox}[breakable, size=fbox, boxrule=1pt, pad at break*=1mm,colback=cellbackground, colframe=cellborder]
\prompt{In}{incolor}{241}{\hspace{4pt}}
\begin{Verbatim}[commandchars=\\\{\}]
\PY{n}{data}\PY{p}{,} \PY{n}{d} \PY{o}{=}  \PY{n}{make\PYZus{}blobs}\PY{p}{(}\PY{n}{n\PYZus{}samples}\PY{o}{=} \PY{l+m+mi}{30000}\PY{p}{,} \PY{n}{n\PYZus{}features}\PY{o}{=} \PY{l+m+mi}{2}\PY{p}{,} \PY{n}{centers} \PY{o}{=} \PY{l+m+mi}{15}\PY{p}{,} \PY{n}{random\PYZus{}state}\PY{o}{=}\PY{l+m+mi}{123}\PY{p}{)}
\end{Verbatim}
\end{tcolorbox}

    \begin{tcolorbox}[breakable, size=fbox, boxrule=1pt, pad at break*=1mm,colback=cellbackground, colframe=cellborder]
\prompt{In}{incolor}{244}{\hspace{4pt}}
\begin{Verbatim}[commandchars=\\\{\}]
\PY{o}{\PYZpc{}\PYZpc{}time}
\PY{n}{model}\PY{p}{,} \PY{n}{cl\PYZus{}model} \PY{o}{=} \PY{n}{myKMeans}\PY{p}{(}\PY{l+m+mi}{15}\PY{p}{,} \PY{n}{data}\PY{p}{,} \PY{l+m+mi}{1}\PY{p}{,} \PY{l+s+s1}{\PYZsq{}}\PY{l+s+s1}{ }\PY{l+s+s1}{\PYZsq{}}\PY{p}{)}
\end{Verbatim}
\end{tcolorbox}

    \begin{Verbatim}[commandchars=\\\{\}]


NUMBER OF CLUSTERS: 15




Визуализация данных
\end{Verbatim}

    \begin{center}
    \adjustimage{max size={0.9\linewidth}{0.9\paperheight}}{output_40_1.png}
    \end{center}
    { \hspace*{\fill} \\}
    
    \begin{Verbatim}[commandchars=\\\{\}]


Визуализация центроидов
\end{Verbatim}

    \begin{center}
    \adjustimage{max size={0.9\linewidth}{0.9\paperheight}}{output_40_3.png}
    \end{center}
    { \hspace*{\fill} \\}
    
    \begin{Verbatim}[commandchars=\\\{\}]


Визуализация к какому классу относятся объкты после инициализации центроидов
\end{Verbatim}

    \begin{center}
    \adjustimage{max size={0.9\linewidth}{0.9\paperheight}}{output_40_5.png}
    \end{center}
    { \hspace*{\fill} \\}
    
    \begin{Verbatim}[commandchars=\\\{\}]


Визуализация после одного обновления положения центроидов
\end{Verbatim}

    \begin{center}
    \adjustimage{max size={0.9\linewidth}{0.9\paperheight}}{output_40_7.png}
    \end{center}
    { \hspace*{\fill} \\}
    
    \begin{Verbatim}[commandchars=\\\{\}]


Визуализация окончательного результата:
\end{Verbatim}

    \begin{center}
    \adjustimage{max size={0.9\linewidth}{0.9\paperheight}}{output_40_9.png}
    \end{center}
    { \hspace*{\fill} \\}
    
    \begin{Verbatim}[commandchars=\\\{\}]
CPU times: user 6min 40s, sys: 2.77 s, total: 6min 43s
Wall time: 12min 14s
\end{Verbatim}

    \begin{tcolorbox}[breakable, size=fbox, boxrule=1pt, pad at break*=1mm,colback=cellbackground, colframe=cellborder]
\prompt{In}{incolor}{243}{\hspace{4pt}}
\begin{Verbatim}[commandchars=\\\{\}]
\PY{o}{\PYZpc{}\PYZpc{}time}
\PY{n}{model}\PY{p}{,} \PY{n}{cl\PYZus{}model} \PY{o}{=} \PY{n}{myKMeans}\PY{p}{(}\PY{l+m+mi}{15}\PY{p}{,} \PY{n}{data}\PY{p}{,} \PY{l+m+mi}{1}\PY{p}{,} \PY{l+s+s1}{\PYZsq{}}\PY{l+s+s1}{KMeans++}\PY{l+s+s1}{\PYZsq{}}\PY{p}{)}
\end{Verbatim}
\end{tcolorbox}

    \begin{Verbatim}[commandchars=\\\{\}]


NUMBER OF CLUSTERS: 15




Визуализация данных
\end{Verbatim}

    \begin{center}
    \adjustimage{max size={0.9\linewidth}{0.9\paperheight}}{output_41_1.png}
    \end{center}
    { \hspace*{\fill} \\}
    
    \begin{Verbatim}[commandchars=\\\{\}]


Визуализация центроидов
\end{Verbatim}

    \begin{center}
    \adjustimage{max size={0.9\linewidth}{0.9\paperheight}}{output_41_3.png}
    \end{center}
    { \hspace*{\fill} \\}
    
    \begin{Verbatim}[commandchars=\\\{\}]


Визуализация к какому классу относятся объкты после инициализации центроидов
\end{Verbatim}

    \begin{center}
    \adjustimage{max size={0.9\linewidth}{0.9\paperheight}}{output_41_5.png}
    \end{center}
    { \hspace*{\fill} \\}
    
    \begin{Verbatim}[commandchars=\\\{\}]


Визуализация после одного обновления положения центроидов
\end{Verbatim}

    \begin{center}
    \adjustimage{max size={0.9\linewidth}{0.9\paperheight}}{output_41_7.png}
    \end{center}
    { \hspace*{\fill} \\}
    
    \begin{Verbatim}[commandchars=\\\{\}]


Визуализация окончательного результата:
\end{Verbatim}

    \begin{center}
    \adjustimage{max size={0.9\linewidth}{0.9\paperheight}}{output_41_9.png}
    \end{center}
    { \hspace*{\fill} \\}
    
    \begin{Verbatim}[commandchars=\\\{\}]
CPU times: user 3min 29s, sys: 1 s, total: 3min 30s
Wall time: 3min 34s
\end{Verbatim}

    \hypertarget{dataset-birch2-ux441ux43aux430ux447ux430ux43d-ux43eux442ux441ux44eux434ux430-httpcs.joensuu.fisipudatasets}{%
\subsubsection{dataset ``birch2'' скачан отсюда:
http://cs.joensuu.fi/sipu/datasets/}\label{dataset-birch2-ux441ux43aux430ux447ux430ux43d-ux43eux442ux441ux44eux434ux430-httpcs.joensuu.fisipudatasets}}

    \begin{tcolorbox}[breakable, size=fbox, boxrule=1pt, pad at break*=1mm,colback=cellbackground, colframe=cellborder]
\prompt{In}{incolor}{245}{\hspace{4pt}}
\begin{Verbatim}[commandchars=\\\{\}]
\PY{n}{data} \PY{o}{=} \PY{n}{np}\PY{o}{.}\PY{n}{loadtxt}\PY{p}{(}\PY{l+s+s2}{\PYZdq{}}\PY{l+s+s2}{/Users/ofirserovlad/Downloads/birch2.txt}\PY{l+s+s2}{\PYZdq{}}\PY{p}{,} \PY{n}{dtype}\PY{o}{=}\PY{l+s+s2}{\PYZdq{}}\PY{l+s+s2}{int}\PY{l+s+s2}{\PYZdq{}}\PY{p}{)}
\end{Verbatim}
\end{tcolorbox}

    \begin{tcolorbox}[breakable, size=fbox, boxrule=1pt, pad at break*=1mm,colback=cellbackground, colframe=cellborder]
\prompt{In}{incolor}{247}{\hspace{4pt}}
\begin{Verbatim}[commandchars=\\\{\}]
\PY{o}{\PYZpc{}\PYZpc{}time}
\PY{n}{model}\PY{p}{,} \PY{n}{cl\PYZus{}model} \PY{o}{=} \PY{n}{myKMeans}\PY{p}{(}\PY{l+m+mi}{2}\PY{p}{,} \PY{n}{data}\PY{p}{,} \PY{l+m+mi}{1}\PY{p}{,} \PY{l+s+s1}{\PYZsq{}}\PY{l+s+s1}{ }\PY{l+s+s1}{\PYZsq{}}\PY{p}{)}
\end{Verbatim}
\end{tcolorbox}

    \begin{Verbatim}[commandchars=\\\{\}]


NUMBER OF CLUSTERS: 2




Визуализация данных
\end{Verbatim}

    \begin{center}
    \adjustimage{max size={0.9\linewidth}{0.9\paperheight}}{output_44_1.png}
    \end{center}
    { \hspace*{\fill} \\}
    
    \begin{Verbatim}[commandchars=\\\{\}]


Визуализация центроидов
\end{Verbatim}

    \begin{center}
    \adjustimage{max size={0.9\linewidth}{0.9\paperheight}}{output_44_3.png}
    \end{center}
    { \hspace*{\fill} \\}
    
    \begin{Verbatim}[commandchars=\\\{\}]


Визуализация к какому классу относятся объкты после инициализации центроидов
\end{Verbatim}

    \begin{center}
    \adjustimage{max size={0.9\linewidth}{0.9\paperheight}}{output_44_5.png}
    \end{center}
    { \hspace*{\fill} \\}
    
    \begin{Verbatim}[commandchars=\\\{\}]


Визуализация после одного обновления положения центроидов
\end{Verbatim}

    \begin{center}
    \adjustimage{max size={0.9\linewidth}{0.9\paperheight}}{output_44_7.png}
    \end{center}
    { \hspace*{\fill} \\}
    
    \begin{Verbatim}[commandchars=\\\{\}]


Визуализация окончательного результата:
\end{Verbatim}

    \begin{center}
    \adjustimage{max size={0.9\linewidth}{0.9\paperheight}}{output_44_9.png}
    \end{center}
    { \hspace*{\fill} \\}
    
    \begin{Verbatim}[commandchars=\\\{\}]
CPU times: user 44.2 s, sys: 336 ms, total: 44.5 s
Wall time: 44.1 s
\end{Verbatim}

    \begin{tcolorbox}[breakable, size=fbox, boxrule=1pt, pad at break*=1mm,colback=cellbackground, colframe=cellborder]
\prompt{In}{incolor}{250}{\hspace{4pt}}
\begin{Verbatim}[commandchars=\\\{\}]
\PY{o}{\PYZpc{}\PYZpc{}time}
\PY{n}{model}\PY{p}{,} \PY{n}{cl\PYZus{}model} \PY{o}{=} \PY{n}{myKMeans}\PY{p}{(}\PY{l+m+mi}{2}\PY{p}{,} \PY{n}{data}\PY{p}{,} \PY{l+m+mi}{1}\PY{p}{,} \PY{l+s+s1}{\PYZsq{}}\PY{l+s+s1}{KMeans++}\PY{l+s+s1}{\PYZsq{}}\PY{p}{)}
\end{Verbatim}
\end{tcolorbox}

    \begin{Verbatim}[commandchars=\\\{\}]


NUMBER OF CLUSTERS: 2




Визуализация данных
\end{Verbatim}

    \begin{center}
    \adjustimage{max size={0.9\linewidth}{0.9\paperheight}}{output_45_1.png}
    \end{center}
    { \hspace*{\fill} \\}
    
    \begin{Verbatim}[commandchars=\\\{\}]


Визуализация центроидов
\end{Verbatim}

    \begin{center}
    \adjustimage{max size={0.9\linewidth}{0.9\paperheight}}{output_45_3.png}
    \end{center}
    { \hspace*{\fill} \\}
    
    \begin{Verbatim}[commandchars=\\\{\}]


Визуализация к какому классу относятся объкты после инициализации центроидов
\end{Verbatim}

    \begin{center}
    \adjustimage{max size={0.9\linewidth}{0.9\paperheight}}{output_45_5.png}
    \end{center}
    { \hspace*{\fill} \\}
    
    \begin{Verbatim}[commandchars=\\\{\}]


Визуализация после одного обновления положения центроидов
\end{Verbatim}

    \begin{center}
    \adjustimage{max size={0.9\linewidth}{0.9\paperheight}}{output_45_7.png}
    \end{center}
    { \hspace*{\fill} \\}
    
    \begin{Verbatim}[commandchars=\\\{\}]


Визуализация окончательного результата:
\end{Verbatim}

    \begin{center}
    \adjustimage{max size={0.9\linewidth}{0.9\paperheight}}{output_45_9.png}
    \end{center}
    { \hspace*{\fill} \\}
    
    \begin{Verbatim}[commandchars=\\\{\}]
CPU times: user 44 s, sys: 490 ms, total: 44.5 s
Wall time: 45.6 s
\end{Verbatim}

    \hypertarget{ux432ux44bux432ux43eux434}{%
\subsection{Вывод:}\label{ux432ux44bux432ux43eux434}}

\hypertarget{ux43fux440ux43eux430ux43dux430ux43bux438ux437ux438ux440ux43eux432ux430ux432-ux432ux441ux435-ux43fux440ux438ux43cux435ux440ux44b-ux43cux43eux436ux43dux43e-ux441ux43aux430ux437ux430ux442ux44c-ux43e-ux442ux43eux43c-ux447ux442ux43e-kmeans-ux432-ux431ux43eux43bux44cux448ux438ux43dux441ux442ux432ux435-ux43fux440ux438ux43cux435ux440ux43eux432-ux440ux430ux431ux43eux442ux430ux435ux442-ux431ux44bux441ux442ux440ux435ux435-ux43cux43eux436ux435ux442-ux437ux430ux432ux438ux441ux435ux442ux44c-ux43eux442-ux43dux430ux447ux430ux43bux44cux43dux43eux439-ux438ux43dux438ux446ux438ux430ux43bux438ux437ux430ux446ux438ux438-ux438-ux43eux442-ux440ux430ux441ux43fux440ux435ux434ux435ux43bux435ux43dux43dux43eux441ux442ux438-ux434ux430ux43dux43dux44bux445}{%
\subsubsection{Проанализировав все примеры, можно сказать о том, что
``kMeans++''" в большинстве примеров работает быстрее (может зависеть от
начальной инициализации и от распределенности
данных)}\label{ux43fux440ux43eux430ux43dux430ux43bux438ux437ux438ux440ux43eux432ux430ux432-ux432ux441ux435-ux43fux440ux438ux43cux435ux440ux44b-ux43cux43eux436ux43dux43e-ux441ux43aux430ux437ux430ux442ux44c-ux43e-ux442ux43eux43c-ux447ux442ux43e-kmeans-ux432-ux431ux43eux43bux44cux448ux438ux43dux441ux442ux432ux435-ux43fux440ux438ux43cux435ux440ux43eux432-ux440ux430ux431ux43eux442ux430ux435ux442-ux431ux44bux441ux442ux440ux435ux435-ux43cux43eux436ux435ux442-ux437ux430ux432ux438ux441ux435ux442ux44c-ux43eux442-ux43dux430ux447ux430ux43bux44cux43dux43eux439-ux438ux43dux438ux446ux438ux430ux43bux438ux437ux430ux446ux438ux438-ux438-ux43eux442-ux440ux430ux441ux43fux440ux435ux434ux435ux43bux435ux43dux43dux43eux441ux442ux438-ux434ux430ux43dux43dux44bux445}}

\hypertarget{ux43cux438ux43dux443ux441ux43eux43c-kmeans-like-ux430ux43bux433ux43eux440ux438ux442ux43cux43eux432-ux44fux432ux43bux44fux435ux442ux441ux44f-ux442ux43e-ux447ux442ux43e-ux43eux43dux438-ux43cux43eux433ux443ux442-ux43dux435-ux43dux430ux445ux43eux434ux438ux442ux44c-ux433ux43bux43eux431ux430ux43bux44cux43dux44bux439-ux43cux438ux43dux438ux43cux443ux43c.-ux442ux430ux43aux436ux435-ux43dux443ux436ux43dux43e-ux43dux430ux445ux43eux434ux438ux442ux44c-ux43aux43eux43bux438ux447ux435ux441ux442ux432ux43e-ux43aux43bux430ux441ux442ux435ux440ux43eux432-ux441ux430ux43cux43eux441ux442ux43eux44fux442ux435ux43bux44cux43dux43e.}{%
\subsubsection{Минусом kMeans-like алгоритмов является то, что они могут
``не находить глобальный минимум''. Также нужно находить количество
кластеров
самостоятельно.}\label{ux43cux438ux43dux443ux441ux43eux43c-kmeans-like-ux430ux43bux433ux43eux440ux438ux442ux43cux43eux432-ux44fux432ux43bux44fux435ux442ux441ux44f-ux442ux43e-ux447ux442ux43e-ux43eux43dux438-ux43cux43eux433ux443ux442-ux43dux435-ux43dux430ux445ux43eux434ux438ux442ux44c-ux433ux43bux43eux431ux430ux43bux44cux43dux44bux439-ux43cux438ux43dux438ux43cux443ux43c.-ux442ux430ux43aux436ux435-ux43dux443ux436ux43dux43e-ux43dux430ux445ux43eux434ux438ux442ux44c-ux43aux43eux43bux438ux447ux435ux441ux442ux432ux43e-ux43aux43bux430ux441ux442ux435ux440ux43eux432-ux441ux430ux43cux43eux441ux442ux43eux44fux442ux435ux43bux44cux43dux43e.}}

\hypertarget{ux43dux430-ux431ux43eux43bux44cux448ux435ux43c-ux43aux43eux43bux438ux447ux435ux441ux442ux432ux435-ux43aux43bux430ux441ux442ux435ux440ux43eux432-ux43fux440ux438-ux442ux43eux43c-ux436ux435-ux43aux43eux43bux438ux447ux435ux441ux442ux432ux435-ux43eux431ux44aux435ux43aux442ux43eux432-ux430ux43bux433ux43eux440ux438ux442ux43c-ux43eux436ux438ux434ux430ux435ux43cux43e-ux440ux430ux431ux43eux442ux430ux435ux442-ux434ux43eux43bux44cux448ux435-ux43aux43eux43bux438ux447ux435ux441ux442ux432ux43e-ux43eux431ux44aux435ux43aux442ux43eux432-ux432ux43bux438ux44fux435ux442-ux442ux430ux43aux436ux435-ux43fux43e-ux43dux430ux43fux440ux430ux432ux43bux435ux43dux438ux44e-ux43dux435ux441ux43aux43eux43bux44cux43aux43e-ux441ux438ux43bux44cux43dux435ux435-ux43fux43e-ux432ux440ux435ux43cux435ux43dux438}{%
\subsubsection{На большем количестве кластеров при том же количестве
объектов алгоритм ожидаемо работает дольше; количество объектов влияет
также ``по направлению'', несколько сильнее ``по
времени''}\label{ux43dux430-ux431ux43eux43bux44cux448ux435ux43c-ux43aux43eux43bux438ux447ux435ux441ux442ux432ux435-ux43aux43bux430ux441ux442ux435ux440ux43eux432-ux43fux440ux438-ux442ux43eux43c-ux436ux435-ux43aux43eux43bux438ux447ux435ux441ux442ux432ux435-ux43eux431ux44aux435ux43aux442ux43eux432-ux430ux43bux433ux43eux440ux438ux442ux43c-ux43eux436ux438ux434ux430ux435ux43cux43e-ux440ux430ux431ux43eux442ux430ux435ux442-ux434ux43eux43bux44cux448ux435-ux43aux43eux43bux438ux447ux435ux441ux442ux432ux43e-ux43eux431ux44aux435ux43aux442ux43eux432-ux432ux43bux438ux44fux435ux442-ux442ux430ux43aux436ux435-ux43fux43e-ux43dux430ux43fux440ux430ux432ux43bux435ux43dux438ux44e-ux43dux435ux441ux43aux43eux43bux44cux43aux43e-ux441ux438ux43bux44cux43dux435ux435-ux43fux43e-ux432ux440ux435ux43cux435ux43dux438}}


    % Add a bibliography block to the postdoc
    
    
    
    \end{document}
